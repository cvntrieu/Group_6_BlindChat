\documentclass[a4paper,12pt]{report}
% --- CẤU HÌNH TIẾNG VIỆT & FONT ---
\usepackage[utf8]{inputenc}
\usepackage[T5]{fontenc}
\usepackage[vietnamese]{babel}
\usepackage{mathptmx} % Font Times New Roman


% --- CẤU HÌNH CĂN LỀ ---
\usepackage{geometry}
\geometry{
    a4paper,
    left=30mm,
    right=20mm,
    top=25mm,
    bottom=25mm
}
\usepackage{setspace}
\onehalfspacing
\setlength{\parskip}{6pt}

% --- CÁC GÓI HỖ TRỢ ---
\usepackage{graphicx}
\usepackage{float}
\usepackage{amssymb} % Cho checkmark symbol
\usepackage{hyperref}
\hypersetup{pdftitle={Báo cáo Tương tác Người Máy}, colorlinks=true, linkcolor=black, urlcolor=blue}
\usepackage{titlesec}
\usepackage{array} 
\usepackage{tikz} % Để vẽ khung
\usetikzlibrary{calc}

% --- SỬA LỖI MÀU SẮC TẠI ĐÂY ---
\usepackage{xcolor}
% Định nghĩa trực tiếp màu Xanh Navy (RGB: 0, 0, 128) để tránh lỗi "Undefined color"
\definecolor{navy}{RGB}{0,0,128} 

% --- ĐỊNH DẠNG TIÊU ĐỀ CHƯƠNG ---
\titleformat{\chapter}[display]
  {\normalfont\bfseries\centering\fontsize{16}{19}\selectfont}
  {\chaptertitlename\ \thechapter}
  {10pt}
  {\fontsize{18}{21}\selectfont}
\titlespacing*{\chapter}{0pt}{-20pt}{20pt}

% =========================================
% BẮT ĐẦU NỘI DUNG
% =========================================
\begin{document}

% --- TRANG BÌA (ĐÃ FIX LỖI TRANG TRẮNG) ---
\begin{titlepage}
    \begin{tikzpicture}[remember picture, overlay]
        \colorlet{FrameColor}{navy} 
        \draw[FrameColor, line width = 3pt, double distance = 3pt, double = white] 
            ($(current page.north west) + (2cm,-2cm)$) rectangle ($(current page.south east) + (-2cm,2cm)$);
        
        % Họa tiết 4 góc
        \draw[FrameColor, line width=2pt] ($(current page.north west) + (2cm,-2cm)$) +(-2pt,8pt) -- +(8pt,8pt) -- +(8pt,-2pt);
        \draw[FrameColor, line width=1.5pt] ($(current page.north west) + (2cm,-2cm)$) +(-2pt,3pt) -- +(3pt,3pt) -- +(3pt,-2pt);
        
        \draw[FrameColor, line width=2pt] ($(current page.north east) + (-2cm,-2cm)$) +(2pt,8pt) -- +(-8pt,8pt) -- +(-8pt,-2pt);
        \draw[FrameColor, line width=1.5pt] ($(current page.north east) + (-2cm,-2cm)$) +(2pt,3pt) -- +(-3pt,3pt) -- +(-3pt,-2pt);
        
        \draw[FrameColor, line width=2pt] ($(current page.south west) + (2cm,2cm)$) +(-2pt,-8pt) -- +(8pt,-8pt) -- +(8pt,2pt);
        \draw[FrameColor, line width=1.5pt] ($(current page.south west) + (2cm,2cm)$) +(-2pt,-3pt) -- +(3pt,-3pt) -- +(3pt,2pt);
        
        \draw[FrameColor, line width=2pt] ($(current page.south east) + (-2cm,2cm)$) +(2pt,-8pt) -- +(-8pt,-8pt) -- +(-8pt,2pt);
        \draw[FrameColor, line width=1.5pt] ($(current page.south east) + (-2cm,2cm)$) +(2pt,-3pt) -- +(-3pt,-3pt) -- +(-3pt,2pt);
    \end{tikzpicture}

    \begin{center}
        \vspace*{-0.5cm} % Kéo nội dung lên cao một chút
        
        {\fontsize{14}{16}\selectfont \textbf{TRƯỜNG ĐẠI HỌC CÔNG NGHỆ}} \\
        \vspace{0.3cm}
        {\fontsize{14}{16}\selectfont \textbf{KHOA CÔNG NGHỆ THÔNG TIN}} \\
        
        \vspace{0.2cm}
        \rule{3cm}{0.5pt} 
        
        \vspace{1cm} 
        \includegraphics[width=3.5cm]{logo-uet.png} 
        \vspace{1cm} 
        
        {\fontsize{18}{24}\selectfont \textbf{BÁO CÁO BÀI TẬP LỚN CUỐI KÌ}} \\
        \vspace{0.3cm}
        {\fontsize{18}{24}\selectfont \textbf{BỘ MÔN}} \\
        \vspace{0.3cm}
        {\fontsize{18}{24}\selectfont \textbf{TƯƠNG TÁC NGƯỜI - MÁY}} \\ 
        \vspace{0.3cm}
        {\fontsize{18}{24}\selectfont \textbf{NĂM HỌC 2024 - 2025}} \\
        
        \vspace{1.2cm} 
        
        {\fontsize{16}{19}\selectfont \textbf{Nhóm 6}} 
        
        \vspace{0.8cm} 
        
        % Danh sách thành viên rút gọn ở bìa để đỡ tốn chỗ
        \begin{tabular}{r l} 
            \textbf{Giảng viên hướng dẫn:} & TS. Ngô Thị Duyên \\
            \textbf{Môn học:} & Tương tác Người - Máy \\
            \textbf{Lớp học phần:} & INT2041 2 \\
            \textbf{Thành viên:} & Nguyễn Văn Hòa - 23021556 \\
                                 & Cao Vũ Nhật Triều - 23021740 \\
                                 & Nguyễn Duy Phong - 23021656 \\
                                 & Hoàng Minh Quân - 22028019 \\
                                 & Đặng Hoàng Minh Nghĩa - 22028054 \\
                                 & Nguyễn Tuấn Dương - 22028230
        \end{tabular}
        
        \vfill
        \hfill 
        \vspace*{0.5cm} % Giảm khoảng cách dưới cùng để tránh tràn trang
    \end{center}
\end{titlepage} 
% LƯU Ý: Đã xóa lệnh \newpage ở đây vì không cần thiết

% --- TRANG ĐÓNG GÓP (NỐI TIẾP NGAY SAU BÌA) ---
% --- TRANG ĐÓNG GÓP (NỐI TIẾP NGAY SAU BÌA) ---
\thispagestyle{empty} 
\begin{center}
    {\fontsize{16}{19}\selectfont \textbf{THÔNG TIN ĐÓNG GÓP THÀNH VIÊN}} \\
    \vspace{1cm}
\end{center}

\begin{table}[h]
    \centering
    \renewcommand{\arraystretch}{1.5} 
    \setlength{\tabcolsep}{5pt}
    
    \begin{tabular}{|p{5cm}|p{7.5cm}|c|}
        \hline
        \centering\textbf{MSV - Thành viên} & \centering\textbf{Nội dung thực hiện} & \textbf{Tỉ lệ} \\
        \hline
        \textbf{23021740} \newline Cao Vũ Nhật Triều & Phát triển Frontend, UI/UX, luồng dữ liệu \& tài khoản & 16.67\% \\
        \hline
        \textbf{23021656} \newline Nguyễn Duy Phong & Phát triển Frontend, UI/UX, luồng dữ liệu \& tài khoản & 16.67\% \\
        \hline
        \textbf{23021556} \newline Nguyễn Văn Hòa & Phát triển AI - Agent & 16.67\% \\
        \hline
        \textbf{22028230} \newline Nguyễn Tuấn Dương & Phát triển Backend, Database & 16.65\% \\
        \hline
        \textbf{22028054} \newline Đặng Hoàng Minh Nghĩa & Phát triển AI - Agent & 16.67\% \\
        \hline
        \textbf{22028219} \newline Hoàng Minh Quân & Phát triển AI - Agent & 16.67\% \\
        \hline
    \end{tabular}
    
    \vspace{0.3cm} % Khoảng cách giữa bảng và chú thích
    % Chú thích bảng (Viết thủ công để đảm bảo hiển thị đúng "Bảng 1")
    \textbf{Bảng 1: Phân chia công việc trong nhóm}
\end{table}

% Đoạn văn ngắn ngay dưới bảng
\indent % Lệnh này để không thụt đầu dòng, giúp văn bản thẳng hàng đẹp
Các thành viên trong nhóm đều tham gia xây dựng ý tưởng và đã cùng nhau phối hợp để giải quyết những vấn đề phát sinh trong quá trình triển khai, đảm bảo tiến độ và chất lượng của dự án.

\newpage

% --- LỜI CAM ĐOAN ---
\thispagestyle{empty}
\begin{center}
    {\fontsize{16}{19}\selectfont \textbf{LỜI CAM ĐOAN}}
    \vspace{1cm}
\end{center}

Chúng em xin cam đoan đây là công trình nghiên cứu của nhóm chúng em thực hiện, dưới sự hướng dẫn của Cán bộ hướng dẫn: TS. Ngô Thị Duyên.

Chúng em xin cam đoan: Nội dung trình bày trong báo cáo bài tập lớn môn học này là trung thực và không sao chép bất kỳ tài liệu hay công trình nghiên cứu nào khác mà không trích dẫn nguồn gốc rõ ràng theo quy định.

Các thông tin, hình ảnh, mã nguồn (nếu có) được trình bày là kết quả tự thân của nhóm trong quá trình thực hiện dự án \textbf{XÂY DỰNG ỨNG DỤNG HỖ TRỢ NGƯỜI KHIẾM THỊ BLINDCHAT}.

\vspace{2cm}
\begin{flushright}
    \textbf{Nhóm sinh viên thực hiện}\\
    \vspace{1cm}
    Nhóm 6
\end{flushright}

\newpage

% --- TÓM TẮT DỰ ÁN ---
\thispagestyle{empty}
\begin{center}
    {\fontsize{16}{19}\selectfont \textbf{TÓM TẮT DỰ ÁN}}
    \vspace{1cm}
\end{center}

Dự án \textbf{BlindChat} được thiết kế nhằm mục tiêu xây dựng một ứng dụng web hỗ trợ người khiếm thị thông qua giao diện giọng nói AI. Ứng dụng cho phép người dùng tương tác với trợ lý ảo bằng giọng nói để thực hiện các tác vụ hàng ngày như đọc và tóm tắt tài liệu PDF, mô tả hình ảnh từ camera, và điều khiển giao diện bằng lệnh thoại.

Về mặt kỹ thuật, ứng dụng được xây dựng theo kiến trúc fullstack hiện đại gồm 3 thành phần chính:

\begin{itemize}
    \item \textbf{Frontend:} Được phát triển bằng Next.js 15 kết hợp với React 19 và LiveKit Client SDK để xử lý giao tiếp real-time giọng nói và video. Giao diện hỗ trợ chat transcript, video streaming, screen sharing và dark/light theme.
    
    \item \textbf{Backend:} Sử dụng ASP.NET Core 9 với Entity Framework Core và SQL Server để quản lý người dùng, lưu trữ lịch sử hội thoại và xác thực JWT Token.
    
    \item \textbf{AI Agent:} Được xây dựng bằng Python với LiveKit Agents SDK, tích hợp OpenAI GPT-4o-mini cho xử lý ngôn ngữ tự nhiên, Azure Speech Services cho Speech-to-Text và Text-to-Speech, cùng khả năng xử lý hình ảnh từ camera qua OpenAI Vision API.
\end{itemize}

Hệ thống cung cấp 4 công cụ chính cho người dùng: (1) Đọc và tóm tắt file PDF từ thư mục Downloads; (2) Mô tả hình ảnh từ camera; (3) Điều khiển UI bằng giọng nói (camera, microphone, chat panel); (4) Truy vấn ngày giờ hiện tại. Lịch sử hội thoại được lưu trữ và đồng bộ với backend để duy trì context liên tục giữa các phiên làm việc.

Dự án hướng tới việc cải thiện khả năng tiếp cận công nghệ (accessibility) cho người khiếm thị, giúp họ tương tác với máy tính và các tài liệu số một cách độc lập và hiệu quả thông qua giao diện giọng nói thân thiện.

\newpage

% --- MỤC LỤC ---
\tableofcontents
\newpage

% --- NỘI DUNG CHÍNH ---
% =========================================
% CHƯƠNG 1: MỞ ĐẦU
% =========================================
\chapter{MỞ ĐẦU}

\section{Đặt vấn đề}

Theo Tổ chức Y tế Thế giới (WHO), trên toàn cầu có khoảng 2.2 tỷ người gặp vấn đề về thị lực, trong đó có ít nhất 1 tỷ người có tình trạng suy giảm thị lực có thể phòng ngừa hoặc chưa được điều trị. Tại Việt Nam, ước tính có khoảng 2 triệu người khiếm thị và mù lòa, chiếm khoảng 2\% dân số. Đây là nhóm đối tượng gặp nhiều khó khăn trong việc tiếp cận thông tin và sử dụng các thiết bị công nghệ hiện đại.

Trong bối cảnh chuyển đổi số đang diễn ra mạnh mẽ, việc tương tác với máy tính và các thiết bị số trở thành nhu cầu thiết yếu của mọi người. Tuy nhiên, phần lớn các ứng dụng và giao diện người dùng hiện nay được thiết kế chủ yếu dựa trên tương tác thị giác (visual interaction) như: màn hình, bàn phím, chuột, touch screen. Điều này tạo ra rào cản lớn đối với người khiếm thị trong việc:

\begin{itemize}
    \item Đọc và xử lý tài liệu số (PDF, Word, email)
    \item Truy cập thông tin trên web và ứng dụng
    \item Nhận diện môi trường xung quanh thông qua camera
    \item Điều khiển các thiết bị và ứng dụng một cách độc lập
\end{itemize}

Các công cụ hỗ trợ hiện có như screen reader (NVDA, JAWS, VoiceOver) tuy hữu ích nhưng vẫn yêu cầu người dùng phải ghi nhớ nhiều phím tắt phức tạp và không thể xử lý các tác vụ yêu cầu ``nhìn'' như mô tả hình ảnh hoặc nhận diện đối tượng.

Với sự phát triển vượt bậc của trí tuệ nhân tạo (AI), đặc biệt là các mô hình ngôn ngữ lớn (Large Language Models - LLM) và xử lý giọng nói, giờ đây có thể xây dựng một hệ thống tương tác hoàn toàn bằng giọng nói, cho phép người khiếm thị giao tiếp với máy tính một cách tự nhiên như đang nói chuyện với một người trợ lý thực sự.

\section{Hiện trạng}

\subsection{Tình hình sử dụng công nghệ của người khiếm thị}

Hiện nay, người khiếm thị tại Việt Nam chủ yếu sử dụng các công cụ hỗ trợ sau để tương tác với thiết bị số:

\begin{enumerate}
    \item \textbf{Screen Reader (Trình đọc màn hình):} Các phần mềm như NVDA (miễn phí), JAWS (có phí), VoiceOver (tích hợp sẵn trên macOS/iOS) giúp đọc nội dung trên màn hình thành giọng nói. Tuy nhiên, người dùng cần phải học và ghi nhớ hàng trăm phím tắt để điều khiển.
    
    \item \textbf{Braille Display:} Thiết bị hiển thị chữ nổi Braille kết nối với máy tính. Chi phí cao (từ 1,000 - 5,000 USD) khiến phần lớn người khiếm thị tại Việt Nam không thể tiếp cận.
    
    \item \textbf{Voice Assistant cơ bản:} Siri, Google Assistant, Alexa cung cấp tương tác giọng nói nhưng chức năng hạn chế, chủ yếu tập trung vào các tác vụ đơn giản như đặt báo thức, tra cứu thời tiết.
\end{enumerate}

\subsection{Những thách thức chính}

\begin{itemize}
    \item \textbf{Rào cản ngôn ngữ:} Hầu hết các công cụ hỗ trợ người khiếm thị được phát triển bằng tiếng Anh, hỗ trợ tiếng Việt còn hạn chế.
    
    \item \textbf{Đường cong học tập cao:} Việc sử dụng screen reader đòi hỏi người dùng phải ghi nhớ nhiều phím tắt phức tạp.
    
    \item \textbf{Không xử lý được nội dung hình ảnh:} Các công cụ hiện tại không thể ``nhìn'' và mô tả hình ảnh, biểu đồ, hoặc môi trường xung quanh.
    
    \item \textbf{Thiếu tính cá nhân hóa:} Không có khả năng hiểu ngữ cảnh và đưa ra phản hồi phù hợp với từng người dùng.
\end{itemize}

\section{Các giải pháp sẵn có}

\subsection{Giải pháp trong nước}

Tại Việt Nam, các giải pháp hỗ trợ người khiếm thị còn khá hạn chế:

\begin{itemize}
    \item \textbf{Ứng dụng Sách nói:} Một số ứng dụng như ``Sách nói Việt Nam'' cung cấp audiobook cho người khiếm thị, nhưng nội dung còn hạn chế và không có tính tương tác.
    
    \item \textbf{NVDA với tiếng Việt:} Phần mềm NVDA được cộng đồng Việt hóa nhưng chất lượng giọng đọc chưa tự nhiên và thiếu nhiều tính năng nâng cao.
    
    \item \textbf{Các dự án nghiên cứu:} Một số đề tài nghiên cứu tại các trường đại học về công nghệ hỗ trợ người khiếm thị nhưng chưa được thương mại hóa và triển khai rộng rãi.
\end{itemize}

\subsection{Giải pháp quốc tế}

\begin{itemize}
    \item \textbf{Be My Eyes:} Ứng dụng kết nối người khiếm thị với tình nguyện viên sáng mắt qua video call để được hỗ trợ mô tả hình ảnh. Gần đây tích hợp GPT-4 Vision để tự động mô tả hình ảnh.
    
    \item \textbf{Seeing AI (Microsoft):} Ứng dụng sử dụng AI để nhận diện và mô tả đối tượng, đọc văn bản, nhận diện khuôn mặt. Chỉ hỗ trợ tiếng Anh và một số ngôn ngữ phổ biến.
    
    \item \textbf{Aira:} Dịch vụ cung cấp ``agent'' là người thật để hỗ trợ người khiếm thị 24/7 qua camera điện thoại. Chi phí cao (99-199 USD/tháng).
    
    \item \textbf{OrCam MyEye:} Thiết bị đeo gắn trên kính, sử dụng AI để đọc văn bản và nhận diện khuôn mặt. Chi phí rất cao (khoảng 4,500 USD).
\end{itemize}

\section{Hạn chế của các giải pháp hiện tại}

Qua phân tích các giải pháp sẵn có, chúng tôi nhận thấy những hạn chế chính sau:

\begin{enumerate}
    \item \textbf{Chi phí cao:} Các giải pháp tiên tiến như OrCam MyEye, Aira có chi phí vượt quá khả năng tài chính của đa số người khiếm thị tại Việt Nam.
    
    \item \textbf{Hạn chế ngôn ngữ:} Phần lớn không hỗ trợ hoặc hỗ trợ tiếng Việt kém.
    
    \item \textbf{Phụ thuộc vào thiết bị chuyên dụng:} Nhiều giải pháp yêu cầu thiết bị phần cứng riêng, không tận dụng được thiết bị sẵn có của người dùng (laptop, smartphone).
    
    \item \textbf{Thiếu tích hợp:} Mỗi ứng dụng giải quyết một vấn đề riêng lẻ, người dùng phải chuyển đổi giữa nhiều ứng dụng để hoàn thành công việc.
    
    \item \textbf{Không có context và memory:} Các voice assistant hiện tại không nhớ ngữ cảnh hội thoại trước đó, mỗi lần tương tác là một phiên mới hoàn toàn.
\end{enumerate}

\section{Mục tiêu và Phạm vi đề tài}

\subsection{Mục tiêu}

Dự án \textbf{BlindChat} được phát triển với các mục tiêu sau:

\begin{enumerate}
    \item \textbf{Xây dựng giao diện giọng nói hoàn toàn:} Cho phép người khiếm thị tương tác với hệ thống 100\% bằng giọng nói, không cần sử dụng bàn phím hay chuột.
    
    \item \textbf{Tích hợp AI đa phương thức:} Kết hợp xử lý ngôn ngữ tự nhiên (NLP), nhận dạng giọng nói (STT), tổng hợp giọng nói (TTS), và xử lý hình ảnh (Vision) trong một hệ thống thống nhất.
    
    \item \textbf{Hỗ trợ đọc tài liệu thông minh:} Cho phép người dùng yêu cầu đọc hoặc tóm tắt các file PDF bằng giọng nói.
    
    \item \textbf{Mô tả hình ảnh từ camera:} Sử dụng AI Vision để mô tả những gì camera nhìn thấy, giúp người khiếm thị ``nhìn'' môi trường xung quanh.
    
    \item \textbf{Duy trì ngữ cảnh hội thoại:} Lưu trữ và sử dụng lịch sử hội thoại để AI hiểu context và đưa ra phản hồi chính xác hơn.
    
    \item \textbf{Thiết kế Accessibility-first:} Áp dụng các nguyên tắc thiết kế accessible từ đầu, đảm bảo trải nghiệm tốt nhất cho người khiếm thị.
\end{enumerate}

\subsection{Phạm vi}

\textbf{Trong phạm vi đề tài:}
\begin{itemize}
    \item Phát triển ứng dụng web với giao diện giọng nói real-time
    \item Tích hợp AI Agent với các chức năng: đọc file, mô tả hình ảnh, điều khiển UI
    \item Xây dựng hệ thống backend lưu trữ lịch sử hội thoại
    \item Hỗ trợ tiếng Anh (có thể mở rộng tiếng Việt trong tương lai)
    \item Chạy trên trình duyệt web (Chrome, Firefox, Edge)
\end{itemize}

\textbf{Ngoài phạm vi đề tài:}
\begin{itemize}
    \item Phát triển ứng dụng mobile native (iOS/Android)
    \item Tích hợp thiết bị phần cứng chuyên dụng
    \item Hỗ trợ đa ngôn ngữ (ngoài tiếng Anh)
    \item Tích hợp với các hệ thống enterprise
\end{itemize}

\section{Đóng góp của đề tài}

Đề tài mang lại những đóng góp sau:

\begin{enumerate}
    \item \textbf{Về mặt khoa học:}
    \begin{itemize}
        \item Nghiên cứu và áp dụng các nguyên tắc thiết kế Human-Computer Interaction (HCI) cho người khiếm thị
        \item Tích hợp nhiều công nghệ AI tiên tiến trong một hệ thống thống nhất
        \item Đề xuất kiến trúc hệ thống voice-first application
    \end{itemize}
    
    \item \textbf{Về mặt thực tiễn:}
    \begin{itemize}
        \item Cung cấp một công cụ miễn phí, dễ tiếp cận cho người khiếm thị
        \item Giảm rào cản công nghệ, giúp người khiếm thị độc lập hơn trong công việc và cuộc sống
        \item Mã nguồn mở, có thể được cộng đồng phát triển và mở rộng
    \end{itemize}
\end{enumerate}

\section{Cấu trúc báo cáo}

Báo cáo được tổ chức thành các chương như sau:

\begin{itemize}
    \item \textbf{Chương 1 - Mở đầu:} Trình bày đặt vấn đề, hiện trạng, các giải pháp sẵn có, mục tiêu và phạm vi đề tài.
    
    \item \textbf{Chương 2 - Nghiên cứu người dùng:} Phân tích đối tượng người dùng, xây dựng User Persona, User Journey Map và xác định các yêu cầu hệ thống.
    
    \item \textbf{Chương 3 - Cơ sở lý thuyết và Công nghệ:} Trình bày nền tảng lý thuyết về HCI, Accessibility và các công nghệ sử dụng.
    
    \item \textbf{Chương 4 - Thiết kế hệ thống:} Mô tả kiến trúc, thiết kế giao diện và luồng tương tác.
    
    \item \textbf{Chương 5 - Triển khai và Kết quả:} Trình bày quá trình triển khai, demo và đánh giá hệ thống.
    
    \item \textbf{Chương 6 - Kết luận:} Tổng kết kết quả đạt được, hạn chế và hướng phát triển.
\end{itemize}

 % Mở đầu và Đặt vấn đề
% =========================================
% CHƯƠNG 2: NGHIÊN CỨU NGƯỜI DÙNG
% =========================================
\chapter{NGHIÊN CỨU NGƯỜI DÙNG}

Trong quá trình phát triển ứng dụng BlindChat, nhóm đã thực hiện nghiên cứu người dùng (User Research) để hiểu rõ nhu cầu, hành vi và những khó khăn mà người khiếm thị gặp phải khi sử dụng công nghệ. Chương này trình bày các phương pháp nghiên cứu, kết quả phân tích và các artifacts UX được xây dựng.

\section{Phương pháp nghiên cứu}

\subsection{Nghiên cứu thứ cấp (Secondary Research)}

Nhóm đã thu thập và phân tích thông tin từ các nguồn sau:

\begin{itemize}
    \item Báo cáo của WHO về tình trạng thị lực toàn cầu
    \item Nghiên cứu về accessibility và assistive technology
    \item Hướng dẫn WCAG 2.1 (Web Content Accessibility Guidelines)
    \item Tài liệu về thiết kế giao diện cho người khiếm thị
    \item Case study từ các sản phẩm: Be My Eyes, Seeing AI, VoiceOver
\end{itemize}

\subsection{Nghiên cứu sơ cấp (Primary Research)}

\begin{itemize}
    \item \textbf{Quan sát gián tiếp:} Xem các video về cách người khiếm thị sử dụng công nghệ trên YouTube
    \item \textbf{Phân tích đánh giá ứng dụng:} Đọc reviews của người dùng khiếm thị trên App Store và Google Play về các ứng dụng hỗ trợ
    \item \textbf{Tham gia cộng đồng:} Theo dõi các diễn đàn và nhóm Facebook của cộng đồng người khiếm thị Việt Nam
\end{itemize}

\section{Phân tích đối tượng người dùng}

\subsection{Đặc điểm chung của người khiếm thị}

Dựa trên nghiên cứu, nhóm xác định các đặc điểm quan trọng của đối tượng người dùng:

\begin{table}[H]
    \centering
    \renewcommand{\arraystretch}{1.3}
    \begin{tabular}{|p{4cm}|p{10cm}|}
        \hline
        \textbf{Đặc điểm} & \textbf{Mô tả} \\
        \hline
        Mức độ khiếm thị & Từ mù hoàn toàn đến thị lực kém (low vision) \\
        \hline
        Độ tuổi & Đa dạng, từ sinh viên đến người cao tuổi \\
        \hline
        Trình độ công nghệ & Từ cơ bản đến nâng cao, đã quen với screen reader \\
        \hline
        Thiết bị sử dụng & Smartphone, laptop, máy tính để bàn \\
        \hline
        Nhu cầu chính & Đọc tài liệu, giao tiếp, học tập, làm việc \\
        \hline
        Rào cản & Giao diện phức tạp, thiếu hỗ trợ giọng nói tiếng Việt \\
        \hline
    \end{tabular}
    \caption{Đặc điểm đối tượng người dùng}
\end{table}

\subsection{Phân khúc người dùng (User Segments)}

\begin{enumerate}
    \item \textbf{Sinh viên khiếm thị:} Cần đọc giáo trình, tài liệu PDF, làm bài tập
    \item \textbf{Nhân viên văn phòng:} Cần xử lý email, tài liệu công việc, họp trực tuyến
    \item \textbf{Người cao tuổi mất thị lực:} Cần giao tiếp, đọc tin tức, sử dụng dịch vụ số
    \item \textbf{Người khiếm thị bẩm sinh:} Đã quen với công nghệ hỗ trợ, cần giải pháp hiệu quả hơn
\end{enumerate}

\section{User Persona}

Dựa trên kết quả nghiên cứu, nhóm xây dựng 2 User Persona đại diện:

\subsection{Persona 1: Minh - Sinh viên đại học}

\begin{table}[H]
    \centering
    \renewcommand{\arraystretch}{1.3}
    \begin{tabular}{|p{4cm}|p{10cm}|}
        \hline
        \multicolumn{2}{|c|}{\textbf{PERSONA: MINH}} \\
        \hline
        \textbf{Thông tin cơ bản} & Nam, 21 tuổi, sinh viên năm 3 ngành CNTT \\
        \hline
        \textbf{Tình trạng thị lực} & Mù hoàn toàn từ nhỏ \\
        \hline
        \textbf{Thiết bị} & MacBook Pro với VoiceOver, iPhone \\
        \hline
        \textbf{Kỹ năng công nghệ} & Cao, thành thạo screen reader và phím tắt \\
        \hline
        \textbf{Mục tiêu} & 
        \begin{itemize}
            \item Đọc giáo trình PDF nhanh chóng
            \item Làm bài tập lập trình độc lập
            \item Tham gia họp nhóm online hiệu quả
        \end{itemize} \\
        \hline
        \textbf{Pain Points} & 
        \begin{itemize}
            \item Tài liệu PDF scan không đọc được bằng screen reader
            \item Không thể xem biểu đồ, hình ảnh trong bài giảng
            \item Mất nhiều thời gian để navigate trong tài liệu dài
        \end{itemize} \\
        \hline
        \textbf{Quote} & ``Tôi muốn có một trợ lý AI hiểu được tôi đang cần gì và giúp tôi nhanh chóng, không cần phải nhớ hàng trăm phím tắt.'' \\
        \hline
    \end{tabular}
    \caption{User Persona - Minh}
\end{table}

\subsection{Persona 2: Lan - Nhân viên văn phòng}

\begin{table}[H]
    \centering
    \renewcommand{\arraystretch}{1.3}
    \begin{tabular}{|p{4cm}|p{10cm}|}
        \hline
        \multicolumn{2}{|c|}{\textbf{PERSONA: LAN}} \\
        \hline
        \textbf{Thông tin cơ bản} & Nữ, 35 tuổi, nhân viên kế toán \\
        \hline
        \textbf{Tình trạng thị lực} & Thị lực kém (low vision) do bệnh tiểu đường \\
        \hline
        \textbf{Thiết bị} & Laptop Windows với NVDA, smartphone Android \\
        \hline
        \textbf{Kỹ năng công nghệ} & Trung bình, đang học sử dụng screen reader \\
        \hline
        \textbf{Mục tiêu} & 
        \begin{itemize}
            \item Xử lý hóa đơn và báo cáo PDF
            \item Giao tiếp với đồng nghiệp qua email
            \item Duy trì công việc độc lập
        \end{itemize} \\
        \hline
        \textbf{Pain Points} & 
        \begin{itemize}
            \item Khó nhớ các phím tắt của NVDA
            \item Không thể đọc số liệu trên hóa đơn scan
            \item Cảm thấy ngại khi phải nhờ đồng nghiệp giúp đỡ
        \end{itemize} \\
        \hline
        \textbf{Quote} & ``Tôi muốn một công cụ đơn giản, chỉ cần nói là được, không cần phải học nhiều thứ phức tạp.'' \\
        \hline
    \end{tabular}
    \caption{User Persona - Lan}
\end{table}

\section{User Journey Map}

\subsection{Journey Map: Đọc tài liệu PDF}

\begin{table}[H]
    \centering
    \renewcommand{\arraystretch}{1.2}
    \small
    \begin{tabular}{|p{2.5cm}|p{3cm}|p{3cm}|p{3cm}|p{3cm}|}
        \hline
        \textbf{Giai đoạn} & \textbf{1. Khởi động} & \textbf{2. Kết nối} & \textbf{3. Yêu cầu} & \textbf{4. Nhận kết quả} \\
        \hline
        \textbf{Hành động} & Mở ứng dụng BlindChat & Bấm ``Start Call'' & Nói: ``Đọc file PDF mới nhất'' & Nghe AI đọc nội dung \\
        \hline
        \textbf{Suy nghĩ} & ``Hy vọng nó hoạt động tốt'' & ``Chờ kết nối...'' & ``Không biết nó hiểu không'' & ``Tuyệt vời, đúng file tôi cần'' \\
        \hline
        \textbf{Cảm xúc} & Tò mò & Hồi hộp & Kỳ vọng & Hài lòng \\
        \hline
        \textbf{Pain Points} & Không biết UI hiện gì & Không biết đã kết nối chưa & Phải nói rõ ràng & File dài, cần tóm tắt \\
        \hline
        \textbf{Cơ hội} & Audio feedback ngay & Thông báo bằng giọng & Xác nhận lại yêu cầu & Tùy chọn tóm tắt/đọc chi tiết \\
        \hline
    \end{tabular}
    \caption{User Journey Map - Đọc tài liệu PDF}
\end{table}

\subsection{Journey Map: Mô tả hình ảnh từ camera}

\begin{table}[H]
    \centering
    \renewcommand{\arraystretch}{1.2}
    \small
    \begin{tabular}{|p{2.5cm}|p{3cm}|p{3cm}|p{3cm}|p{3cm}|}
        \hline
        \textbf{Giai đoạn} & \textbf{1. Bật camera} & \textbf{2. Hướng camera} & \textbf{3. Hỏi AI} & \textbf{4. Nhận mô tả} \\
        \hline
        \textbf{Hành động} & Nói: ``Bật camera'' & Di chuyển thiết bị & Nói: ``Mô tả những gì bạn thấy'' & Nghe mô tả chi tiết \\
        \hline
        \textbf{Suy nghĩ} & ``Camera đã bật chưa?'' & ``Không biết hướng có đúng không'' & ``Nó có thấy rõ không?'' & ``À, vậy đây là...'' \\
        \hline
        \textbf{Cảm xúc} & Lo lắng & Bất an & Hồi hộp & Vui mừng \\
        \hline
        \textbf{Pain Points} & Không biết camera có hoạt động & Không có feedback về góc nhìn & Chờ đợi & Mô tả có thể không đủ chi tiết \\
        \hline
        \textbf{Cơ hội} & Xác nhận bằng âm thanh & Hướng dẫn bằng giọng & Phản hồi nhanh & Cho phép hỏi thêm \\
        \hline
    \end{tabular}
    \caption{User Journey Map - Mô tả hình ảnh}
\end{table}

\section{Yêu cầu hệ thống (Requirements)}

Dựa trên kết quả nghiên cứu người dùng, nhóm xác định các yêu cầu hệ thống như sau:

\subsection{Yêu cầu chức năng (Functional Requirements)}

\begin{table}[H]
    \centering
    \renewcommand{\arraystretch}{1.3}
    \begin{tabular}{|c|p{10cm}|c|}
        \hline
        \textbf{ID} & \textbf{Mô tả} & \textbf{Độ ưu tiên} \\
        \hline
        FR01 & Người dùng có thể tương tác hoàn toàn bằng giọng nói & Cao \\
        \hline
        FR02 & Hệ thống có thể đọc và tóm tắt file PDF & Cao \\
        \hline
        FR03 & Hệ thống có thể mô tả hình ảnh từ camera & Cao \\
        \hline
        FR04 & Người dùng có thể điều khiển UI bằng giọng nói (bật/tắt camera, mic) & Trung bình \\
        \hline
        FR05 & Hệ thống lưu trữ lịch sử hội thoại để duy trì context & Trung bình \\
        \hline
        FR06 & Người dùng có thể xem transcript của hội thoại & Thấp \\
        \hline
        FR07 & Hệ thống hỗ trợ screen sharing & Thấp \\
        \hline
    \end{tabular}
    \caption{Yêu cầu chức năng}
\end{table}

\subsection{Yêu cầu phi chức năng (Non-Functional Requirements)}

\begin{table}[H]
    \centering
    \renewcommand{\arraystretch}{1.3}
    \begin{tabular}{|c|p{10cm}|c|}
        \hline
        \textbf{ID} & \textbf{Mô tả} & \textbf{Độ ưu tiên} \\
        \hline
        NFR01 & Thời gian phản hồi của AI < 3 giây & Cao \\
        \hline
        NFR02 & Giao diện phải tuân thủ WCAG 2.1 Level AA & Cao \\
        \hline
        NFR03 & Hệ thống phải cung cấp audio feedback cho mọi hành động & Cao \\
        \hline
        NFR04 & Hỗ trợ các trình duyệt phổ biến (Chrome, Firefox, Edge) & Trung bình \\
        \hline
        NFR05 & Bảo mật dữ liệu người dùng với JWT authentication & Trung bình \\
        \hline
        NFR06 & Hệ thống hoạt động ổn định với 100 người dùng đồng thời & Thấp \\
        \hline
    \end{tabular}
    \caption{Yêu cầu phi chức năng}
\end{table}

\section{Nguyên tắc thiết kế Accessibility}

Dựa trên nghiên cứu người dùng và các tiêu chuẩn accessibility, nhóm áp dụng các nguyên tắc thiết kế sau:

\subsection{Nguyên tắc POUR (WCAG 2.1)}

\begin{enumerate}
    \item \textbf{Perceivable (Có thể nhận thức):}
    \begin{itemize}
        \item Mọi thông tin đều có thể được nhận thức qua thính giác
        \item Cung cấp text alternative cho nội dung non-text
        \item Hỗ trợ screen reader hoàn toàn
    \end{itemize}
    
    \item \textbf{Operable (Có thể vận hành):}
    \begin{itemize}
        \item Điều khiển hoàn toàn bằng giọng nói
        \item Hỗ trợ keyboard navigation
        \item Không có giới hạn thời gian cho tương tác
    \end{itemize}
    
    \item \textbf{Understandable (Có thể hiểu):}
    \begin{itemize}
        \item Ngôn ngữ đơn giản, rõ ràng
        \item Phản hồi nhất quán và có thể dự đoán
        \item Hướng dẫn sử dụng bằng giọng nói
    \end{itemize}
    
    \item \textbf{Robust (Bền vững):}
    \begin{itemize}
        \item Tương thích với các công nghệ hỗ trợ
        \item Semantic HTML đúng chuẩn
        \item ARIA labels đầy đủ
    \end{itemize}
\end{enumerate}

\subsection{Nguyên tắc Voice-First Design}

\begin{enumerate}
    \item \textbf{Audio Feedback Liên tục:} Mọi hành động và trạng thái đều được thông báo bằng âm thanh
    
    \item \textbf{Confirmation trước Action quan trọng:} Xác nhận lại trước khi thực hiện các hành động quan trọng
    
    \item \textbf{Error Recovery dễ dàng:} Cho phép người dùng sửa lỗi bằng giọng nói đơn giản
    
    \item \textbf{Context Awareness:} AI nhớ ngữ cảnh để người dùng không phải lặp lại thông tin
    
    \item \textbf{Progressive Disclosure:} Cung cấp thông tin từ tổng quan đến chi tiết theo yêu cầu
\end{enumerate}

\section{Kết luận chương}

Qua quá trình nghiên cứu người dùng, nhóm đã:

\begin{itemize}
    \item Hiểu rõ đặc điểm, nhu cầu và khó khăn của người khiếm thị khi sử dụng công nghệ
    \item Xây dựng được 2 User Persona đại diện cho các phân khúc người dùng chính
    \item Thiết lập User Journey Map cho các use case quan trọng
    \item Xác định được các yêu cầu chức năng và phi chức năng của hệ thống
    \item Đề xuất các nguyên tắc thiết kế accessibility phù hợp
\end{itemize}

Các kết quả nghiên cứu này sẽ là nền tảng để thiết kế và phát triển ứng dụng BlindChat trong các chương tiếp theo.

 % Nghiên cứu người dùng (User Research)
% =========================================
% CHƯƠNG 3: CƠ SỞ LÝ THUYẾT VÀ CÔNG NGHỆ
% =========================================
\chapter{CƠ SỞ LÝ THUYẾT VÀ CÔNG NGHỆ}

Chương này trình bày các cơ sở lý thuyết về tương tác người-máy (HCI), accessibility, và các công nghệ được sử dụng để xây dựng ứng dụng BlindChat.

\section{Cơ sở lý thuyết về Tương tác Người-Máy}

\subsection{Khái niệm HCI}

Tương tác Người-Máy (Human-Computer Interaction - HCI) là lĩnh vực nghiên cứu về thiết kế, đánh giá và triển khai các hệ thống máy tính tương tác để con người sử dụng, cùng với việc nghiên cứu các hiện tượng xung quanh chúng.

HCI tập trung vào ba yếu tố chính:
\begin{itemize}
    \item \textbf{Con người (Human):} Người dùng với các đặc điểm nhận thức, vật lý và xã hội
    \item \textbf{Máy tính (Computer):} Các thiết bị và hệ thống công nghệ
    \item \textbf{Tương tác (Interaction):} Cách thức con người và máy tính giao tiếp với nhau
\end{itemize}

\subsection{Mô hình tương tác}

Trong HCI, có nhiều mô hình tương tác khác nhau:

\begin{enumerate}
    \item \textbf{Command Line Interface (CLI):} Tương tác thông qua dòng lệnh văn bản
    \item \textbf{Graphical User Interface (GUI):} Tương tác thông qua giao diện đồ họa với chuột và bàn phím
    \item \textbf{Voice User Interface (VUI):} Tương tác thông qua giọng nói
    \item \textbf{Natural Language Interface (NLI):} Tương tác bằng ngôn ngữ tự nhiên
    \item \textbf{Multimodal Interface:} Kết hợp nhiều phương thức tương tác
\end{enumerate}

BlindChat sử dụng \textbf{Voice User Interface (VUI)} kết hợp với \textbf{Natural Language Interface} để cho phép người dùng tương tác bằng giọng nói tự nhiên.

\subsection{Nguyên tắc thiết kế giao diện}

\subsubsection{10 Heuristics của Nielsen}

Jakob Nielsen đề xuất 10 nguyên tắc heuristic cho thiết kế giao diện:

\begin{enumerate}
    \item \textbf{Visibility of system status:} Hệ thống luôn thông báo trạng thái cho người dùng
    \item \textbf{Match between system and real world:} Sử dụng ngôn ngữ quen thuộc với người dùng
    \item \textbf{User control and freedom:} Cho phép người dùng hủy bỏ và quay lại
    \item \textbf{Consistency and standards:} Nhất quán trong thiết kế
    \item \textbf{Error prevention:} Ngăn ngừa lỗi xảy ra
    \item \textbf{Recognition rather than recall:} Giảm tải cho trí nhớ người dùng
    \item \textbf{Flexibility and efficiency of use:} Linh hoạt cho cả người mới và chuyên gia
    \item \textbf{Aesthetic and minimalist design:} Thiết kế tối giản
    \item \textbf{Help users recognize and recover from errors:} Hỗ trợ khắc phục lỗi
    \item \textbf{Help and documentation:} Cung cấp tài liệu hướng dẫn
\end{enumerate}

\subsubsection{Áp dụng cho Voice Interface}

Trong BlindChat, các nguyên tắc này được áp dụng như sau:

\begin{table}[H]
    \centering
    \renewcommand{\arraystretch}{1.3}
    \begin{tabular}{|p{4cm}|p{10cm}|}
        \hline
        \textbf{Nguyên tắc} & \textbf{Áp dụng trong BlindChat} \\
        \hline
        System status & AI phản hồi bằng giọng nói xác nhận đã nhận lệnh \\
        \hline
        Real world match & Sử dụng ngôn ngữ tự nhiên, không cần lệnh đặc biệt \\
        \hline
        User control & Có thể ngắt AI bất cứ lúc nào bằng cách nói \\
        \hline
        Consistency & Cùng một loại yêu cầu luôn có phản hồi tương tự \\
        \hline
        Error prevention & AI xác nhận lại trước khi thực hiện hành động quan trọng \\
        \hline
        Recognition & AI gợi ý các hành động có thể thực hiện \\
        \hline
        Flexibility & Chấp nhận nhiều cách diễn đạt khác nhau \\
        \hline
        Minimalist & Phản hồi ngắn gọn, đi thẳng vào vấn đề \\
        \hline
        Error recovery & Hướng dẫn cách sửa khi hiểu sai ý người dùng \\
        \hline
        Help & Có thể hỏi ``Bạn có thể làm gì?'' để nhận hướng dẫn \\
        \hline
    \end{tabular}
    \caption{Áp dụng Nielsen's Heuristics trong BlindChat}
\end{table}

\section{Accessibility và Universal Design}

\subsection{Khái niệm Accessibility}

Accessibility (Khả năng tiếp cận) trong công nghệ đề cập đến việc thiết kế sản phẩm, thiết bị, dịch vụ hoặc môi trường để mọi người, bao gồm cả người khuyết tật, có thể sử dụng được.

\subsection{Web Content Accessibility Guidelines (WCAG)}

WCAG là bộ tiêu chuẩn quốc tế về accessibility cho nội dung web, được phát triển bởi W3C (World Wide Web Consortium).

\subsubsection{Bốn nguyên tắc POUR}

\begin{enumerate}
    \item \textbf{Perceivable (Có thể nhận thức):}
    \begin{itemize}
        \item Cung cấp text alternatives cho nội dung non-text
        \item Cung cấp alternatives cho time-based media
        \item Nội dung có thể được trình bày theo nhiều cách
        \item Phân biệt foreground và background rõ ràng
    \end{itemize}
    
    \item \textbf{Operable (Có thể vận hành):}
    \begin{itemize}
        \item Tất cả chức năng có thể truy cập từ keyboard
        \item Cho người dùng đủ thời gian để đọc và sử dụng nội dung
        \item Không thiết kế nội dung gây co giật
        \item Giúp người dùng navigate và tìm nội dung
    \end{itemize}
    
    \item \textbf{Understandable (Có thể hiểu):}
    \begin{itemize}
        \item Text content có thể đọc và hiểu được
        \item Nội dung xuất hiện và hoạt động theo cách có thể dự đoán
        \item Giúp người dùng tránh và sửa lỗi
    \end{itemize}
    
    \item \textbf{Robust (Bền vững):}
    \begin{itemize}
        \item Tương thích tối đa với các user agents hiện tại và tương lai
        \item Tương thích với assistive technologies
    \end{itemize}
\end{enumerate}

\subsubsection{Các mức độ tuân thủ WCAG}

\begin{itemize}
    \item \textbf{Level A:} Mức tối thiểu, bắt buộc
    \item \textbf{Level AA:} Mức khuyến nghị, được yêu cầu bởi nhiều luật pháp
    \item \textbf{Level AAA:} Mức cao nhất, tối ưu
\end{itemize}

BlindChat hướng tới tuân thủ \textbf{WCAG 2.1 Level AA}.

\subsection{ARIA (Accessible Rich Internet Applications)}

ARIA là một bộ thuộc tính HTML giúp cải thiện accessibility cho nội dung web động và các widget phức tạp.

Các thuộc tính ARIA quan trọng:
\begin{itemize}
    \item \texttt{aria-label}: Mô tả ngắn gọn cho element
    \item \texttt{aria-describedby}: Liên kết đến element chứa mô tả chi tiết
    \item \texttt{aria-live}: Thông báo cập nhật động cho screen reader
    \item \texttt{role}: Định nghĩa vai trò của element
\end{itemize}

\section{Công nghệ xử lý giọng nói}

\subsection{Speech-to-Text (STT)}

Speech-to-Text là công nghệ chuyển đổi giọng nói thành văn bản. Các phương pháp chính:

\begin{enumerate}
    \item \textbf{Traditional ASR:} Sử dụng Hidden Markov Models (HMM) và acoustic models
    \item \textbf{Deep Learning ASR:} Sử dụng mạng neural như RNN, LSTM, Transformer
    \item \textbf{End-to-End ASR:} Mô hình học trực tiếp từ audio đến text
\end{enumerate}

BlindChat sử dụng \textbf{Azure Speech Services} với các ưu điểm:
\begin{itemize}
    \item Độ chính xác cao với nhiều ngôn ngữ
    \item Real-time streaming transcription
    \item Noise cancellation tích hợp
    \item Low latency (< 500ms)
\end{itemize}

\subsection{Text-to-Speech (TTS)}

Text-to-Speech là công nghệ tổng hợp giọng nói từ văn bản. Các phương pháp:

\begin{enumerate}
    \item \textbf{Concatenative TTS:} Ghép nối các đơn vị âm thanh được ghi âm sẵn
    \item \textbf{Parametric TTS:} Tạo giọng nói từ các tham số acoustic
    \item \textbf{Neural TTS:} Sử dụng deep learning (WaveNet, Tacotron, VITS)
\end{enumerate}

Azure TTS cung cấp:
\begin{itemize}
    \item Neural voices với chất lượng tự nhiên
    \item Hỗ trợ SSML để kiểm soát prosody
    \item Multiple voice styles (cheerful, sad, angry, etc.)
    \item Real-time streaming synthesis
\end{itemize}

\subsection{Voice Activity Detection (VAD)}

VAD là công nghệ phát hiện khi nào có giọng nói trong audio stream. BlindChat sử dụng \textbf{Silero VAD}:

\begin{itemize}
    \item Lightweight model (< 1MB)
    \item High accuracy (> 99\%)
    \item Low latency detection
    \item Configurable silence duration threshold
\end{itemize}

\section{Large Language Models (LLM)}

\subsection{Kiến trúc Transformer}

Transformer là kiến trúc neural network được giới thiệu trong paper ``Attention is All You Need'' (2017). Đặc điểm chính:

\begin{itemize}
    \item \textbf{Self-Attention Mechanism:} Cho phép model xem xét toàn bộ input sequence
    \item \textbf{Positional Encoding:} Mã hóa vị trí của tokens
    \item \textbf{Multi-Head Attention:} Nhiều attention heads học các patterns khác nhau
    \item \textbf{Feed-Forward Networks:} Xử lý output của attention layers
\end{itemize}

\subsection{GPT Models}

GPT (Generative Pre-trained Transformer) là họ models của OpenAI:

\begin{itemize}
    \item \textbf{GPT-3.5:} 175 billion parameters, training data đến 2021
    \item \textbf{GPT-4:} Multimodal, cải thiện reasoning và instruction following
    \item \textbf{GPT-4o-mini:} Phiên bản nhỏ gọn, tối ưu cho latency và cost
\end{itemize}

BlindChat sử dụng \textbf{GPT-4o-mini} vì:
\begin{itemize}
    \item Cân bằng giữa chất lượng và tốc độ phản hồi
    \item Chi phí hợp lý cho ứng dụng real-time
    \item Hỗ trợ function calling/tool use
    \item Multimodal: xử lý cả text và image
\end{itemize}

\subsection{Function Calling / Tool Use}

Function Calling cho phép LLM gọi các functions được định nghĩa sẵn:

\begin{verbatim}
@function_tool
async def process_file_request(ctx: RunContext, 
                                user_input: str) -> str:
    """
    Finds and processes a PDF file based on 
    a user's spoken request.
    """
    # Implementation
\end{verbatim}

Lợi ích:
\begin{itemize}
    \item LLM có thể thực hiện các hành động cụ thể
    \item Structured output thay vì free-form text
    \item Tích hợp với external APIs và services
\end{itemize}

\section{Computer Vision}

\subsection{Vision Language Models}

Vision Language Models (VLM) là các models có khả năng hiểu cả hình ảnh và ngôn ngữ. Ứng dụng trong BlindChat:

\begin{itemize}
    \item Mô tả hình ảnh từ camera
    \item Trả lời câu hỏi về nội dung hình ảnh
    \item Đọc text trong hình ảnh (OCR)
\end{itemize}

BlindChat sử dụng \textbf{GPT-4o Vision API}:
\begin{verbatim}
result = client.chat.completions.create(
    model="gpt-4o-mini",
    messages=[{
        "role": "user",
        "content": [
            {"type": "text", "text": user_input},
            {
                "type": "image_url",
                "image_url": {
                    "url": f"data:image/jpeg;base64,{base64_image}"
                }
            }
        ]
    }],
    max_tokens=100
)
\end{verbatim}

\section{Real-time Communication}

\subsection{WebRTC}

WebRTC (Web Real-Time Communication) là công nghệ cho phép truyền tải audio, video và data trực tiếp giữa các browsers mà không cần plugin.

Các thành phần chính:
\begin{itemize}
    \item \textbf{MediaStream:} Capture audio/video từ microphone và camera
    \item \textbf{RTCPeerConnection:} Thiết lập và quản lý peer-to-peer connection
    \item \textbf{RTCDataChannel:} Truyền tải arbitrary data
\end{itemize}

\subsection{LiveKit}

LiveKit là một open-source platform cho real-time communication, được xây dựng trên WebRTC.

\subsubsection{LiveKit Server}

\begin{itemize}
    \item Selective Forwarding Unit (SFU) architecture
    \item Scalable room management
    \item Token-based authentication
    \item Support simulcast và adaptive bitrate
\end{itemize}

\subsubsection{LiveKit Agents SDK}

LiveKit Agents SDK cho phép xây dựng AI agents tương tác real-time:

\begin{itemize}
    \item \textbf{Voice Pipeline:} STT $\rightarrow$ LLM $\rightarrow$ TTS integration
    \item \textbf{Function Tools:} Define custom tools cho agent
    \item \textbf{Room Events:} Handle participant join/leave
    \item \textbf{Track Management:} Manage audio/video tracks
\end{itemize}

\section{Kiến trúc Backend}

\subsection{ASP.NET Core}

ASP.NET Core là framework cross-platform cho xây dựng web applications và APIs:

\begin{itemize}
    \item High performance
    \item Dependency injection built-in
    \item Middleware pipeline
    \item Support RESTful APIs
\end{itemize}

\subsection{Entity Framework Core}

EF Core là Object-Relational Mapper (ORM) cho .NET:

\begin{itemize}
    \item Code-first approach
    \item LINQ queries
    \item Migrations management
    \item Multiple database providers
\end{itemize}

\subsection{JWT Authentication}

JSON Web Token (JWT) được sử dụng cho authentication:

\begin{verbatim}
{
  "header": {
    "alg": "HS256",
    "typ": "JWT"
  },
  "payload": {
    "sub": "user_id",
    "name": "username",
    "exp": 1234567890
  },
  "signature": "..."
}
\end{verbatim}

\section{Frontend Technologies}

\subsection{Next.js}

Next.js là React framework với các tính năng:

\begin{itemize}
    \item Server-side rendering (SSR)
    \item Static site generation (SSG)
    \item API routes
    \item File-based routing
    \item Automatic code splitting
\end{itemize}

\subsection{React}

React là JavaScript library cho building UI:

\begin{itemize}
    \item Component-based architecture
    \item Virtual DOM for efficient updates
    \item Hooks for state management
    \item Large ecosystem
\end{itemize}

\subsection{TailwindCSS}

TailwindCSS là utility-first CSS framework:

\begin{itemize}
    \item Rapid prototyping
    \item Consistent design system
    \item Responsive design utilities
    \item Dark mode support
\end{itemize}

\section{Kết luận chương}

Chương này đã trình bày:

\begin{itemize}
    \item Cơ sở lý thuyết về HCI và các nguyên tắc thiết kế giao diện
    \item Tiêu chuẩn Accessibility WCAG và ARIA
    \item Các công nghệ xử lý giọng nói: STT, TTS, VAD
    \item Large Language Models và function calling
    \item Computer Vision cho image understanding
    \item Real-time communication với WebRTC và LiveKit
    \item Backend và Frontend technologies
\end{itemize}

Các công nghệ này được kết hợp để xây dựng một hệ thống voice-first application hoàn chỉnh, được trình bày chi tiết trong chương tiếp theo.

 % Cơ sở lý thuyết và Công nghệ
% =========================================
% CHƯƠNG 4: THIẾT KẾ HỆ THỐNG
% =========================================
\chapter{THIẾT KẾ HỆ THỐNG}

Chương này trình bày chi tiết kiến trúc hệ thống, thiết kế cơ sở dữ liệu, thiết kế API và thiết kế giao diện người dùng của ứng dụng BlindChat.

\section{Kiến trúc tổng thể}

\subsection{Tổng quan kiến trúc}

Hệ thống BlindChat được thiết kế theo kiến trúc microservices với 3 thành phần chính:

\begin{figure}[H]
    \centering
    \begin{verbatim}
    +-------------------------------------------------------------+
    |                         Frontend                            |
    |              (Next.js 15 + React 19 + LiveKit)              |
    |         Voice Interface, Video Streaming, Chat UI           |
    +--------------------------+----------------------------------+
                               |
                   +-----------+-----------+
                   |                       |
                   v                       v
    +----------------------+   +--------------------------------+
    |      AI Agent        |   |         Backend API            |
    |  (Python + LiveKit)  |   |   (ASP.NET Core 9 + SQL)       |
    |                      |   |                                |
    | * Voice Assistant    |<--|  * User Authentication         |
    | * OpenAI GPT-4o-mini |   |  * Conversation History        |
    | * Azure STT/TTS      |   |  * Message Storage             |
    | * Vision Processing  |   |                                |
    +----------------------+   +--------------------------------+
    \end{verbatim}
    \caption{Kiến trúc tổng thể hệ thống BlindChat}
\end{figure}

\subsection{Luồng dữ liệu}

\subsubsection{Luồng Voice Interaction}

\begin{enumerate}
    \item User nói vào microphone $\rightarrow$ Frontend capture audio
    \item Audio được stream qua LiveKit Room đến AI Agent
    \item AI Agent sử dụng Azure STT để chuyển speech thành text
    \item Text được gửi đến OpenAI GPT-4o-mini để xử lý
    \item GPT-4o-mini quyết định gọi tool hoặc trả lời trực tiếp
    \item Response text được gửi qua Azure TTS để tạo audio
    \item Audio response được stream về Frontend qua LiveKit
    \item User nghe phản hồi
\end{enumerate}

\subsubsection{Luồng lưu trữ Conversation}

\begin{enumerate}
    \item Mỗi tin nhắn (user/agent) được lưu vào ConversationCache
    \item Khi đủ 5 cặp tin nhắn, cache tự động flush lên Backend
    \item Backend lưu messages vào SQL Server
    \item Khi user reconnect, Agent lấy lịch sử từ Backend để load context
\end{enumerate}

\section{Thiết kế AI Agent}

\subsection{Class Diagram}

\begin{figure}[H]
    \centering
    \begin{verbatim}
    +-----------------------------------------------------+
    |                    Assistant                        |
    |                   (extends Agent)                   |
    +-----------------------------------------------------+
    | - context_pairs: List[Dict]                         |
    | - cache: ConversationCache                          |
    +-----------------------------------------------------+
    | + update_context(username: str)                     |
    | + get_current_date_and_time() -> str                |
    | + process_file_request(user_input: str) -> str      |
    | + describe_camera_view(user_input: str) -> str      |
    | + control_ui_device(target, status) -> str          |
    +-----------------------------------------------------+
                              |
                              | uses
                              v
    +-----------------------------------------------------+
    |                ConversationCache                    |
    +-----------------------------------------------------+
    | - username: str                                     |
    | - pairs_to_flush: int                               |
    | - _pending_messages: List[Dict]                     |
    | - _token: Optional[str]                             |
    +-----------------------------------------------------+
    | + add_user_message(content: str)                    |
    | + add_agent_message(content: str)                   |
    | + flush() -> List[Dict]                             |
    | + get_last_n_pairs(n: int) -> List[Dict]            |
    | + get_history_messages() -> List[Dict]              |
    +-----------------------------------------------------+
    \end{verbatim}
    \caption{Class Diagram của AI Agent}
\end{figure}

\subsection{Function Tools}

AI Agent được trang bị 4 function tools:

\begin{table}[H]
    \centering
    \renewcommand{\arraystretch}{1.5}
    \footnotesize
    \begin{tabular}{|>{\raggedright\arraybackslash}p{3.8cm}|>{\raggedright\arraybackslash}p{4.2cm}|>{\raggedright\arraybackslash}p{5.5cm}|}
        \hline
        \textbf{Tool Name} & \textbf{Trigger Phrases} & \textbf{Chức năng} \\
        \hline
        \texttt{get\_current\_} \texttt{date\_and\_time} & ``What time is it?'', ``What's the date?'' & Trả về ngày giờ hiện tại \\
        \hline
        \texttt{process\_file\_} \texttt{request} & ``Read my PDF'', ``Summarize the report'' & Đọc và tóm tắt file PDF từ Downloads \\
        \hline
        \texttt{describe\_} \texttt{camera\_view} & ``What do you see?'', ``Describe this'' & Mô tả hình ảnh từ camera \\
        \hline
        \texttt{control\_ui\_} \texttt{device} & ``Turn on camera'', ``Mute microphone'' & Điều khiển UI (camera, mic, chat) \\
        \hline
    \end{tabular}
    \caption{Function Tools của AI Agent}
\end{table}

\subsection{Request Routing}

Hệ thống sử dụng LLM để route user request:

\begin{verbatim}
RequestType:
  - request_type: "read raw text" | "read file and summary" | "unsupported"
  - confidence_score: float (0-1)
  - description: str
  - file_name: Optional[str]
  - nth_file: Optional[int]
\end{verbatim}

Luồng routing:
\begin{enumerate}
    \item User input + conversation context $\rightarrow$ Router LLM
    \item Router phân loại request type với confidence score
    \item Nếu confidence $\geq$ 0.7: thực hiện action tương ứng
    \item Nếu confidence $<$ 0.7: trả về ``unsupported''
\end{enumerate}

\section{Thiết kế Backend API}

\subsection{Database Schema}

\begin{figure}[H]
    \centering
    \begin{verbatim}
    +------------------+
    |      User        |
    +------------------+
    | Id: string (PK)  |
    | UserName: string |
    | (IdentityUser)   |
    +--------+---------+
             | 1
             |
             | 1
    +--------v-----------------+
    |   ConversationHistory    |
    +--------------------------+
    | Id: int (PK)             |
    | UserId: string (FK)      |
    | CreatedAt: DateTime      |
    +--------+-----------------+
             | 1
             |
             | *
    +--------v---------+
    |     Message      |
    +------------------+
    | Id: int (PK)     |
    | ConversationId   |
    | SenderType: int  |
    | Content: string  |
    | CreatedAt        |
    +------------------+
    
    SenderType: 0 = User, 1 = Bot
    \end{verbatim}
    \caption{Database Schema}
\end{figure}

\subsection{API Endpoints}

\subsubsection{Account Controller}

\begin{table}[H]
    \centering
    \renewcommand{\arraystretch}{1.3}
    \begin{tabular}{|l|l|p{5cm}|p{4cm}|}
        \hline
        \textbf{Method} & \textbf{Endpoint} & \textbf{Request Body} & \textbf{Response} \\
        \hline
        POST & /api/account/register & \texttt{\{username: string\}} & User + JWT Token \\
        \hline
        POST & /api/account/login & \texttt{\{username: string\}} & User + JWT Token \\
        \hline
    \end{tabular}
    \caption{Account API Endpoints}
\end{table}

\subsubsection{Message Controller}

\begin{table}[H]
    \centering
    \renewcommand{\arraystretch}{1.3}
    \begin{tabular}{|l|l|p{5cm}|p{4cm}|}
        \hline
        \textbf{Method} & \textbf{Endpoint} & \textbf{Request Body} & \textbf{Response} \\
        \hline
        POST & /api/messages & List of CreateMessageDto & List of MessageDto \\
        \hline
    \end{tabular}
    \caption{Message API Endpoint}
\end{table}

\textbf{CreateMessageDto:}
\begin{verbatim}
{
    "senderType": 0 | 1,
    "content": "string",
    "createdAt": "2025-01-01T00:00:00Z"
}
\end{verbatim}

\subsubsection{Conversation History Controller}

\begin{table}[H]
    \centering
    \renewcommand{\arraystretch}{1.3}
    \begin{tabular}{|l|l|p{5cm}|p{4cm}|}
        \hline
        \textbf{Method} & \textbf{Endpoint} & \textbf{Query Params} & \textbf{Response} \\
        \hline
        GET & /api/conversation-history & \texttt{limit: int} & ConversationHistoryDto \\
        \hline
    \end{tabular}
    \caption{Conversation History API Endpoint}
\end{table}

\subsection{Authentication Flow}

\begin{enumerate}
    \item AI Agent gọi POST /api/account/login với username
    \item Backend trả về JWT token
    \item Agent lưu token và sử dụng trong header: \texttt{Authorization: Bearer <token>}
    \item Token có thời hạn 1 giờ, agent tự động refresh khi hết hạn
\end{enumerate}

\section{Thiết kế Frontend}

\subsection{Component Architecture}

\begin{verbatim}
App
|-- SessionProvider (Context: LiveKit Room, AppConfig)
|   +-- ViewController
|       |-- WelcomeView (Before connect)
|       |   +-- StartButton
|       +-- SessionView (After connect)
|           |-- TileLayout (Video tiles)
|           |-- ChatTranscript (Chat history)
|           +-- AgentControlBar
|               |-- TrackToggle (Mic)
|               |-- TrackToggle (Camera)
|               |-- ChatInput
|               +-- LeaveButton
\end{verbatim}

\subsection{State Management}

Frontend sử dụng React Context và LiveKit hooks:

\begin{itemize}
    \item \textbf{SessionContext:} Quản lý connection state, app config
    \item \textbf{useRoomContext:} Access LiveKit Room object
    \item \textbf{useChat:} Manage chat messages
    \item \textbf{useTranscriptions:} Real-time transcriptions
    \item \textbf{useChatMessages:} Merge transcriptions + chat messages
\end{itemize}

\subsection{Connection Flow}

\begin{enumerate}
    \item User click ``Start Call''
    \item Frontend gọi POST /api/connection-details
    \item Server tạo LiveKit access token với:
    \begin{itemize}
        \item Room name: \texttt{voice\_assistant\_room\_\{random\}}
        \item Participant identity: user identity
        \item Grants: roomJoin, canPublish, canSubscribe
    \end{itemize}
    \item Frontend connect đến LiveKit server với token
    \item AI Agent tự động join room khi có participant
\end{enumerate}

\subsection{Data Channel Communication}

Để điều khiển UI từ AI Agent, sử dụng LiveKit Data Channel:

\textbf{Agent gửi command:}
\begin{verbatim}
payload = {
    "type": "control_camera" | "control_microphone" | "control_chat",
    "status": "on" | "off"
}
await room.local_participant.publish_data(
    json.dumps(payload),
    destination_identities=[participant.identity]
)
\end{verbatim}

\textbf{Frontend nhận và xử lý:}
\begin{verbatim}
room.on('dataReceived', (payload, participant) => {
    const command = JSON.parse(payload);
    if (command.type === 'control_camera') {
        toggleCamera(command.status === 'on');
    }
    // ...
});
\end{verbatim}

\section{Thiết kế giao diện}

\subsection{Nguyên tắc thiết kế Voice-First UI}

\begin{enumerate}
    \item \textbf{Minimal Visual Elements:} Giao diện tối giản, không gây phân tán
    \item \textbf{High Contrast:} Độ tương phản cao cho người low vision
    \item \textbf{Large Touch Targets:} Nút bấm lớn, dễ tương tác
    \item \textbf{Audio Feedback:} Mọi hành động đều có phản hồi âm thanh
    \item \textbf{Screen Reader Compatible:} Tương thích hoàn toàn với VoiceOver/NVDA
\end{enumerate}

\subsection{Layout Design}

\subsubsection{Welcome View}

\begin{verbatim}
+-----------------------------------------+
|                                         |
|              [Logo BlindChat]           |
|                                         |
|        "BlindChat Voice Agent"          |
|   "A voice agent support blind users"   |
|                                         |
|          +-----------------+            |
|          |   Start Call    |            |
|          +-----------------+            |
|                                         |
|            [Theme Toggle]               |
|                                         |
+-----------------------------------------+
\end{verbatim}

\subsubsection{Session View}

\begin{verbatim}
+-----------------------------------------+
|  +---------------------------------+    |
|  |                                 |    |
|  |      Video Tile (Agent/User)    |    |
|  |                                 |    |
|  +---------------------------------+    |
|                                         |
|  +---------------------------------+    |
|  | Chat Transcript                 |    |
|  | User: "Read my PDF"             |    |
|  | Agent: "Reading file..."        |    |
|  +---------------------------------+    |
|                                         |
|  +---------------------------------+    |
|  | [Mic] [Camera] [Chat] [Leave]   |    |
|  | [          Text Input         ] |    |
|  +---------------------------------+    |
+-----------------------------------------+
\end{verbatim}

\subsection{Accessibility Features}

\begin{table}[H]
    \centering
    \renewcommand{\arraystretch}{1.3}
    \begin{tabular}{|p{4cm}|p{10cm}|}
        \hline
        \textbf{Feature} & \textbf{Implementation} \\
        \hline
        Keyboard Navigation & Tất cả elements có thể focus bằng Tab \\
        \hline
        ARIA Labels & Mọi button và control đều có aria-label \\
        \hline
        Screen Reader & Semantic HTML, role attributes \\
        \hline
        Color Contrast & Tối thiểu 4.5:1 ratio \\
        \hline
        Focus Indicators & Visible focus ring cho keyboard users \\
        \hline
        Responsive Text & Hỗ trợ zoom đến 200\% \\
        \hline
        Dark Mode & Giảm strain cho người low vision \\
        \hline
    \end{tabular}
    \caption{Accessibility Features}
\end{table}

\section{Sequence Diagrams}

\subsection{Đọc file PDF}

\begin{verbatim}
User          Frontend      LiveKit      AI Agent       Backend
 |               |             |            |              |
 |--"Read PDF"-->|             |            |              |
 |               |--Audio----->|            |              |
 |               |             |--Stream--->|              |
 |               |             |            |--STT         |
 |               |             |            |--GPT-4o------|
 |               |             |            |  (tool call) |
 |               |             |            |--Read file   |
 |               |             |            |--Summarize   |
 |               |             |            |--TTS         |
 |               |             |<--Stream---|              |
 |               |<--Audio-----|            |              |
 |<--Response----|             |            |              |
\end{verbatim}

\subsection{Mô tả camera}

\begin{verbatim}
User          Frontend      LiveKit      AI Agent      OpenAI Vision
 |               |             |            |              |
 |--"What see?"->|             |            |              |
 |               |--Audio+Video>|           |              |
 |               |             |--Stream--->|              |
 |               |             |            |--Capture frame
 |               |             |            |--Base64 encode
 |               |             |            |------------->|
 |               |             |            |<--Description|
 |               |             |            |--TTS         |
 |               |             |<--Stream---|              |
 |<--Description-|             |            |              |
\end{verbatim}

\section{Kết luận chương}

Chương này đã trình bày chi tiết thiết kế hệ thống BlindChat bao gồm:

\begin{itemize}
    \item Kiến trúc tổng thể với 3 thành phần: Frontend, Backend, AI Agent
    \item Thiết kế AI Agent với 4 function tools
    \item Thiết kế Backend API với authentication và message storage
    \item Thiết kế Frontend với component architecture và state management
    \item Thiết kế giao diện tuân thủ accessibility guidelines
    \item Sequence diagrams cho các use case chính
\end{itemize}

Thiết kế này đảm bảo hệ thống có thể mở rộng, bảo trì dễ dàng và đáp ứng các yêu cầu accessibility cho người khiếm thị.
 % Thiết kế hệ thống
% =========================================
% CHƯƠNG 5: TRIỂN KHAI VÀ KẾT QUẢ
% =========================================
\chapter{TRIỂN KHAI VÀ KẾT QUẢ}

Chương này trình bày quá trình triển khai hệ thống BlindChat, các kết quả đạt được và demo các tính năng chính.

\section{Môi trường phát triển}

\subsection{Công cụ và phần mềm}

\begin{table}[H]
    \centering
    \renewcommand{\arraystretch}{1.3}
    \begin{tabular}{|l|l|l|}
        \hline
        \textbf{Công cụ} & \textbf{Phiên bản} & \textbf{Mục đích} \\
        \hline
        Visual Studio Code & 1.95+ & IDE chính cho Frontend và AI Agent \\
        \hline
        Visual Studio 2022 & 17.8+ & IDE cho Backend (.NET) \\
        \hline
        Node.js & 20.x LTS & Runtime cho Frontend \\
        \hline
        Python & 3.11+ & Runtime cho AI Agent \\
        \hline
        .NET SDK & 9.0 & Runtime cho Backend \\
        \hline
        SQL Server & Express 2022 & Database \\
        \hline
        Git & 2.40+ & Version control \\
        \hline
        pnpm & 9.x & Package manager cho Frontend \\
        \hline
        uv & Latest & Package manager cho Python \\
        \hline
    \end{tabular}
    \caption{Công cụ phát triển}
\end{table}

\subsection{Cấu hình hệ thống}

\textbf{Yêu cầu phần cứng tối thiểu:}
\begin{itemize}
    \item CPU: Intel Core i5 hoặc tương đương
    \item RAM: 8GB (khuyến nghị 16GB)
    \item Ổ cứng: 10GB trống
    \item Microphone và Camera (cho testing)
\end{itemize}

\textbf{Hệ điều hành:}
\begin{itemize}
    \item Windows 10/11 (đã test)
    \item macOS 12+ (tương thích)
    \item Linux Ubuntu 22.04+ (tương thích)
\end{itemize}

\section{Cài đặt và cấu hình}

\subsection{Backend Setup}

\begin{verbatim}
# 1. Di chuyển đến thư mục backend
cd backend/api

# 2. Cấu hình connection string trong appsettings.json
{
  "ConnectionStrings": {
    "DefaultConnection": "Data Source=localhost\\SQLEXPRESS;
                          Initial Catalog=BlindChatDb;
                          Integrated Security=True;"
  }
}

# 3. Restore packages và build
dotnet restore
dotnet build

# 4. Chạy migrations
dotnet ef database update

# 5. Chạy server
dotnet watch run
\end{verbatim}

Backend server sẽ chạy tại \texttt{http://localhost:5000}.

\subsection{Frontend Setup}

\begin{verbatim}
# 1. Di chuyển đến thư mục frontend
cd frontend

# 2. Cài đặt dependencies
pnpm install

# 3. Tạo file .env.local
LIVEKIT_API_KEY=your_api_key
LIVEKIT_API_SECRET=your_api_secret
LIVEKIT_URL=wss://your-livekit-server

# 4. Chạy development server
pnpm dev
\end{verbatim}

Frontend sẽ chạy tại \texttt{http://localhost:3000}.

\subsection{AI Agent Setup}

\begin{verbatim}
# 1. Di chuyển đến thư mục ai-agent
cd ai-agent

# 2. Cài đặt dependencies với uv
uv sync

# 3. Tạo file .env
OPENAI_API_KEY=your_openai_key
AZURE_SPEECH_KEY=your_azure_speech_key
AZURE_SPEECH_REGION=eastus
BACKEND_BASE_URL=http://localhost:5000
AGENT_LOGIN_USERNAME=agent_user
LIVEKIT_URL=wss://your-livekit-server
LIVEKIT_API_KEY=your_api_key
LIVEKIT_API_SECRET=your_api_secret

# 4. Chạy agent
python agent.py dev
\end{verbatim}

\section{Triển khai các tính năng}

\subsection{Voice Interaction}

\subsubsection{Speech-to-Text Pipeline}

Cấu hình Azure STT trong AI Agent:

\begin{verbatim}
session = AgentSession(
    stt=azure.STT(
        speech_key=os.environ.get("AZURE_SPEECH_KEY"),
        speech_region=os.environ.get("AZURE_SPEECH_REGION"),
    ),
    # ...
)
\end{verbatim}

\textbf{Kết quả đạt được:}
\begin{itemize}
    \item Độ chính xác nhận dạng: $>$ 95\% với tiếng Anh rõ ràng
    \item Latency: $<$ 500ms từ khi nói xong đến khi nhận được text
    \item Hỗ trợ continuous recognition
\end{itemize}

\subsubsection{Text-to-Speech Pipeline}

\begin{verbatim}
session = AgentSession(
    # ...
    tts=azure.TTS(
        speech_key=os.environ.get("AZURE_SPEECH_KEY"),
        speech_region=os.environ.get("AZURE_SPEECH_REGION"),
    ),
    # ...
)
\end{verbatim}

\textbf{Kết quả đạt được:}
\begin{itemize}
    \item Giọng nói neural tự nhiên
    \item Latency: $<$ 300ms từ text đến audio
    \item Streaming audio cho response dài
\end{itemize}

\subsection{File Reading \& Summarization}

\subsubsection{Implementation}

\begin{verbatim}
def read_file_content(filepath: str, max_chars: int = 8000) -> str:
    ext = os.path.splitext(filepath)[1].lower()
    
    if ext == ".pdf":
        reader = PdfReader(filepath)
        text_content = ""
        for page in reader.pages:
            text_content += page.extract_text() or ""
            if len(text_content) > max_chars:
                break
        return text_content.strip()
    
    elif ext == ".docx":
        doc = Document(filepath)
        text_content = ""
        for para in doc.paragraphs:
            text_content += para.text + "\n"
        return text_content.strip()
    
    # Handle other formats...
\end{verbatim}

\textbf{Kết quả đạt được:}
\begin{itemize}
    \item Hỗ trợ định dạng: PDF, DOCX, TXT, MD, CSV, JSON
    \item Tự động tìm file mới nhất trong Downloads
    \item Tóm tắt với độ dài tùy chỉnh (mặc định 50 words)
    \item Xử lý file lớn (cắt ở 8000 ký tự để tránh quá tải)
\end{itemize}

\subsubsection{Demo Use Cases}

\begin{table}[H]
    \centering
    \renewcommand{\arraystretch}{1.3}
    \begin{tabular}{|p{5cm}|p{9cm}|}
        \hline
        \textbf{User Input} & \textbf{Agent Response} \\
        \hline
        ``Read my latest PDF'' & Đọc nội dung file PDF mới nhất trong Downloads \\
        \hline
        ``Summarize the report'' & Tóm tắt nội dung file và đọc summary \\
        \hline
        ``Read the first PDF'' & Đọc file PDF cũ nhất (theo thời gian modified) \\
        \hline
    \end{tabular}
    \caption{Demo File Reading}
\end{table}

\subsection{Camera Vision}

\subsubsection{Implementation}

\begin{verbatim}
@function_tool
async def describe_camera_view(self, ctx: RunContext, 
                                user_input: str) -> str:
    # Get video track from participant
    video_stream = rtc.VideoStream(video_track_pub.track)
    event = await asyncio.wait_for(anext(video_stream), timeout=2.0)
    frame = event.frame
    
    # Convert frame to image
    rgb_frame = frame.convert(rtc.VideoBufferType.RGB24)
    img = Image.frombytes("RGB", 
                          (rgb_frame.width, rgb_frame.height), 
                          rgb_frame.data)
    
    # Encode to base64
    buf = io.BytesIO()
    img.save(buf, format="JPEG")
    base64_image = base64.b64encode(buf.getvalue()).decode("utf-8")
    
    # Call OpenAI Vision API
    return handle_image_description(user_input, base64_image)
\end{verbatim}

\textbf{Kết quả đạt được:}
\begin{itemize}
    \item Capture frame real-time từ video stream
    \item Mô tả chi tiết objects, người, văn bản trong hình
    \item Response time: 2-4 giây cho mỗi image
    \item Hỗ trợ follow-up questions về cùng hình ảnh
\end{itemize}

\subsubsection{Demo Use Cases}

\begin{table}[H]
    \centering
    \renewcommand{\arraystretch}{1.3}
    \begin{tabular}{|p{5cm}|p{9cm}|}
        \hline
        \textbf{User Input} & \textbf{Mô tả Response} \\
        \hline
        ``What do you see?'' & Mô tả tổng quan cảnh vật trong camera \\
        \hline
        ``Describe my face'' & Mô tả đặc điểm khuôn mặt người dùng \\
        \hline
        ``Read the text'' & OCR - đọc văn bản hiển thị trong camera \\
        \hline
        ``Is anyone there?'' & Phát hiện và mô tả người trong khung hình \\
        \hline
    \end{tabular}
    \caption{Demo Camera Vision}
\end{table}

\subsection{UI Control}

\subsubsection{Implementation}

\begin{verbatim}
@function_tool
async def control_ui_device(
    self, 
    ctx: RunContext, 
    target: Literal["camera", "microphone", "chat"], 
    status: Literal["on", "off"]
) -> str:
    room = get_job_context().room
    participant = next(iter(room.remote_participants.values()), None)
    
    payload = {
        "type": f"control_{target}",
        "status": status
    }
    await room.local_participant.publish_data(
        json.dumps(payload),
        destination_identities=[participant.identity]
    )
    
    return f"Command to turn {target} {status} has been sent."
\end{verbatim}

\textbf{Kết quả đạt được:}
\begin{itemize}
    \item Điều khiển camera: bật/tắt video track
    \item Điều khiển microphone: mute/unmute audio track
    \item Điều khiển chat panel: show/hide chat transcript
    \item Latency: $<$ 100ms từ lệnh đến thực thi
\end{itemize}

\subsection{Conversation History}

\subsubsection{Implementation}

Cache và sync với backend:

\begin{verbatim}
class ConversationCache:
    def add_agent_message(self, content: str):
        dto = {
            "senderType": 1,
            "content": content,
            "createdAt": datetime.now(timezone.utc).isoformat()
        }
        self._pending_messages.append(dto)
        self._pair_count += 1
        
        # Auto flush when threshold reached
        if self._pair_count >= self.pairs_to_flush:
            self.flush()
\end{verbatim}

\textbf{Kết quả đạt được:}
\begin{itemize}
    \item Lưu trữ tin nhắn theo cặp (user + bot)
    \item Tự động sync sau mỗi 5 cặp
    \item Load context khi user reconnect
    \item Persist data trong SQL Server
\end{itemize}

\section{Kết quả Demo}

\subsection{Screenshots}

\textit{(Phần này sẽ chèn các screenshots thực tế của ứng dụng)}

\begin{enumerate}
    \item \textbf{Welcome Screen:} Màn hình chào với nút ``Start Call''
    \item \textbf{Session View:} Giao diện trong phiên voice chat
    \item \textbf{Chat Transcript:} Hiển thị lịch sử hội thoại real-time
    \item \textbf{Control Bar:} Thanh điều khiển mic, camera, chat
\end{enumerate}

\subsection{Performance Metrics}

\begin{table}[H]
    \centering
    \renewcommand{\arraystretch}{1.3}
    \begin{tabular}{|l|c|c|}
        \hline
        \textbf{Metric} & \textbf{Target} & \textbf{Actual} \\
        \hline
        STT Latency & $<$ 500ms & 300-450ms \\
        \hline
        TTS Latency & $<$ 500ms & 200-350ms \\
        \hline
        LLM Response Time & $<$ 3s & 1-2.5s \\
        \hline
        Vision API Response & $<$ 5s & 2-4s \\
        \hline
        End-to-end Response & $<$ 5s & 2-4s \\
        \hline
        UI Control Latency & $<$ 200ms & 50-100ms \\
        \hline
    \end{tabular}
    \caption{Performance Metrics}
\end{table}

\subsection{Accessibility Compliance}

\begin{table}[H]
    \centering
    \renewcommand{\arraystretch}{1.3}
    \begin{tabular}{|l|c|p{6cm}|}
        \hline
        \textbf{Tiêu chí WCAG} & \textbf{Status} & \textbf{Ghi chú} \\
        \hline
        Keyboard Navigation & \checkmark & Tất cả controls có thể focus \\
        \hline
        Screen Reader & \checkmark & Test với NVDA và VoiceOver \\
        \hline
        Color Contrast & \checkmark & Ratio $>$ 4.5:1 \\
        \hline
        Focus Indicators & \checkmark & Visible focus ring \\
        \hline
        ARIA Labels & \checkmark & Đầy đủ cho buttons và inputs \\
        \hline
        Text Resize & \checkmark & Hỗ trợ zoom 200\% \\
        \hline
    \end{tabular}
    \caption{Accessibility Compliance Checklist}
\end{table}

\section{Testing}

\subsection{Unit Testing}

Test các function utilities:

\begin{verbatim}
# test_anything.py
from utils import find_recent_pdfs_in_downloads

def test_find_pdfs():
    pdfs = find_recent_pdfs_in_downloads(days=7)
    assert isinstance(pdfs, list)
    for pdf in pdfs:
        assert "file_name" in pdf
        assert "full_path" in pdf
\end{verbatim}

\subsection{Integration Testing}

Test API endpoints:

\begin{verbatim}
# test_api.py
import requests

def test_login():
    response = requests.post(
        "http://localhost:5000/api/account/login",
        json={"username": "test_user"}
    )
    assert response.status_code == 200
    assert "Token" in response.json()["User"]
\end{verbatim}

\subsection{Manual Testing}

Các test cases được thực hiện thủ công:

\begin{enumerate}
    \item \textbf{Voice Interaction:}
    \begin{itemize}
        \item Nói các câu lệnh khác nhau
        \item Test với giọng nói có nhiễu
        \item Test interrupt (nói khi AI đang trả lời)
    \end{itemize}
    
    \item \textbf{File Reading:}
    \begin{itemize}
        \item Test với PDF nhiều trang
        \item Test với file scan (image-based PDF)
        \item Test với file không tồn tại
    \end{itemize}
    
    \item \textbf{Camera Vision:}
    \begin{itemize}
        \item Test trong điều kiện ánh sáng khác nhau
        \item Test với nhiều objects trong frame
        \item Test OCR với văn bản tiếng Việt/Anh
    \end{itemize}
    
    \item \textbf{Accessibility:}
    \begin{itemize}
        \item Test với NVDA screen reader
        \item Test với VoiceOver (macOS)
        \item Test keyboard-only navigation
    \end{itemize}
\end{enumerate}

\section{Đánh giá hệ thống}

\subsection{Ưu điểm}

\begin{enumerate}
    \item \textbf{Tương tác tự nhiên:} Người dùng có thể giao tiếp bằng ngôn ngữ tự nhiên, không cần học các lệnh đặc biệt
    
    \item \textbf{Đa chức năng:} Tích hợp nhiều tính năng hữu ích trong một ứng dụng: đọc file, vision, điều khiển UI
    
    \item \textbf{Real-time:} Phản hồi nhanh với latency thấp, tạo trải nghiệm mượt mà
    
    \item \textbf{Context-aware:} Lưu trữ và sử dụng lịch sử hội thoại để hiểu ngữ cảnh
    
    \item \textbf{Accessible:} Tuân thủ WCAG 2.1, tương thích với screen readers
    
    \item \textbf{Extensible:} Kiến trúc cho phép dễ dàng thêm function tools mới
\end{enumerate}

\subsection{Hạn chế}

\begin{enumerate}
    \item \textbf{Phụ thuộc Internet:} Yêu cầu kết nối mạng ổn định để hoạt động
    
    \item \textbf{Chi phí API:} Sử dụng các dịch vụ trả phí (OpenAI, Azure)
    
    \item \textbf{Ngôn ngữ:} Hiện chỉ hỗ trợ tốt tiếng Anh
    
    \item \textbf{PDF scan:} Không đọc được PDF dạng hình ảnh (cần OCR riêng)
    
    \item \textbf{Offline mode:} Chưa hỗ trợ chế độ offline
\end{enumerate}

\section{Kết luận chương}

Chương này đã trình bày:

\begin{itemize}
    \item Môi trường phát triển và cấu hình hệ thống
    \item Hướng dẫn cài đặt và chạy các thành phần
    \item Chi tiết triển khai các tính năng chính
    \item Kết quả demo và performance metrics
    \item Accessibility compliance và testing
    \item Đánh giá ưu điểm và hạn chế của hệ thống
\end{itemize}

Hệ thống BlindChat đã được triển khai thành công với đầy đủ các tính năng đề ra, đáp ứng các yêu cầu về chức năng và accessibility cho người khiếm thị.

 % Triển khai và Kết quả
% =========================================
% CHƯƠNG 6: KẾT LUẬN VÀ HƯỚNG PHÁT TRIỂN
% =========================================
\chapter{KẾT LUẬN VÀ HƯỚNG PHÁT TRIỂN}

\section{Tổng kết kết quả đạt được}

\subsection{Về mặt sản phẩm}

Dự án BlindChat đã hoàn thành việc xây dựng một ứng dụng web hỗ trợ người khiếm thị với các tính năng chính:

\begin{enumerate}
    \item \textbf{Giao diện giọng nói hoàn toàn:}
    \begin{itemize}
        \item Người dùng có thể tương tác 100\% bằng giọng nói
        \item Không yêu cầu ghi nhớ phím tắt hay lệnh phức tạp
        \item Phản hồi bằng giọng nói tự nhiên, dễ hiểu
    \end{itemize}
    
    \item \textbf{Đọc và tóm tắt tài liệu:}
    \begin{itemize}
        \item Hỗ trợ đọc file PDF, DOCX và các định dạng văn bản
        \item Tự động tìm file trong thư mục Downloads
        \item Tóm tắt nội dung dài thành ngắn gọn
    \end{itemize}
    
    \item \textbf{Mô tả hình ảnh từ camera:}
    \begin{itemize}
        \item Capture và phân tích hình ảnh real-time
        \item Mô tả chi tiết đối tượng, người, văn bản
        \item Hỗ trợ OCR cho văn bản trong hình ảnh
    \end{itemize}
    
    \item \textbf{Điều khiển giao diện:}
    \begin{itemize}
        \item Bật/tắt camera, microphone bằng giọng nói
        \item Hiển thị/ẩn chat panel
        \item Phản hồi tức thì với latency thấp
    \end{itemize}
    
    \item \textbf{Lưu trữ ngữ cảnh:}
    \begin{itemize}
        \item Duy trì lịch sử hội thoại giữa các phiên
        \item AI hiểu và sử dụng context để phản hồi chính xác hơn
        \item Đồng bộ dữ liệu với backend server
    \end{itemize}
\end{enumerate}

\subsection{Về mặt kỹ thuật}

\begin{enumerate}
    \item \textbf{Kiến trúc hệ thống:}
    \begin{itemize}
        \item Thiết kế microservices với 3 thành phần độc lập
        \item API RESTful chuẩn với authentication JWT
        \item Real-time communication qua WebRTC/LiveKit
    \end{itemize}
    
    \item \textbf{Tích hợp AI:}
    \begin{itemize}
        \item Kết hợp thành công LLM, STT, TTS, Vision trong một pipeline
        \item Function calling cho phép AI thực hiện actions cụ thể
        \item Xử lý đa phương thức (text, audio, image)
    \end{itemize}
    
    \item \textbf{Performance:}
    \begin{itemize}
        \item End-to-end response time: 2-4 giây
        \item STT latency: < 500ms
        \item UI control latency: < 100ms
    \end{itemize}
    
    \item \textbf{Accessibility:}
    \begin{itemize}
        \item Tuân thủ WCAG 2.1 Level AA
        \item Tương thích với screen readers (NVDA, VoiceOver)
        \item Keyboard navigation hoàn toàn
    \end{itemize}
\end{enumerate}

\subsection{Về mặt học thuật}

Dự án đã áp dụng và nghiên cứu các kiến thức:

\begin{enumerate}
    \item \textbf{Human-Computer Interaction (HCI):}
    \begin{itemize}
        \item Nguyên tắc thiết kế giao diện của Nielsen
        \item Voice User Interface design patterns
        \item Accessibility và Universal Design
    \end{itemize}
    
    \item \textbf{User Experience (UX):}
    \begin{itemize}
        \item User Research methodology
        \item Persona và Journey Map creation
        \item Usability testing với target users
    \end{itemize}
    
    \item \textbf{Công nghệ AI:}
    \begin{itemize}
        \item Large Language Models và prompt engineering
        \item Speech processing (STT/TTS)
        \item Computer Vision và multimodal AI
    \end{itemize}
\end{enumerate}

\section{Đóng góp của đề tài}

\subsection{Đóng góp về mặt khoa học}

\begin{enumerate}
    \item Đề xuất và triển khai kiến trúc Voice-First Application cho người khiếm thị
    \item Tích hợp thành công multiple AI services trong real-time voice pipeline
    \item Áp dụng các nguyên tắc HCI vào thiết kế giao diện giọng nói
    \item Nghiên cứu và đánh giá các tiêu chuẩn accessibility (WCAG, ARIA)
\end{enumerate}

\subsection{Đóng góp về mặt thực tiễn}

\begin{enumerate}
    \item Cung cấp công cụ miễn phí, mã nguồn mở cho cộng đồng người khiếm thị
    \item Giảm rào cản công nghệ cho người khuyết tật thị giác
    \item Nền tảng có thể mở rộng thêm nhiều tính năng hỗ trợ
    \item Tài liệu hướng dẫn chi tiết cho việc phát triển tiếp
\end{enumerate}

\section{Hạn chế của đề tài}

\subsection{Hạn chế về kỹ thuật}

\begin{enumerate}
    \item \textbf{Phụ thuộc Internet:}
    \begin{itemize}
        \item Ứng dụng yêu cầu kết nối mạng ổn định
        \item Không có chế độ offline
        \item Latency phụ thuộc vào chất lượng mạng
    \end{itemize}
    
    \item \textbf{Chi phí API:}
    \begin{itemize}
        \item Sử dụng các dịch vụ trả phí (OpenAI, Azure)
        \item Chi phí tăng theo số lượng requests
        \item Cần cân nhắc khi scale lên nhiều users
    \end{itemize}
    
    \item \textbf{Hỗ trợ ngôn ngữ:}
    \begin{itemize}
        \item Hiện chỉ hỗ trợ tốt tiếng Anh
        \item STT/TTS tiếng Việt chưa được tối ưu
        \item Một số file tiếng Việt có thể không đọc đúng
    \end{itemize}
\end{enumerate}

\subsection{Hạn chế về nghiên cứu}

\begin{enumerate}
    \item \textbf{User Research:}
    \begin{itemize}
        \item Chưa có điều kiện phỏng vấn trực tiếp người khiếm thị
        \item Dựa chủ yếu vào secondary research
        \item Cần nhiều user testing thực tế hơn
    \end{itemize}
    
    \item \textbf{Usability Testing:}
    \begin{itemize}
        \item Chưa thực hiện được A/B testing
        \item Số lượng test users còn hạn chế
        \item Cần đánh giá dài hạn hơn
    \end{itemize}
\end{enumerate}

\section{Hướng phát triển}

\subsection{Ngắn hạn (3-6 tháng)}

\begin{enumerate}
    \item \textbf{Cải thiện hỗ trợ tiếng Việt:}
    \begin{itemize}
        \item Tích hợp Vietnamese STT/TTS tốt hơn
        \item Prompt engineering cho tiếng Việt
        \item UI labels và audio feedback tiếng Việt
    \end{itemize}
    
    \item \textbf{Mở rộng file support:}
    \begin{itemize}
        \item Hỗ trợ PDF scan với OCR
        \item Đọc file Excel, PowerPoint
        \item Xử lý file từ cloud storage (Google Drive, OneDrive)
    \end{itemize}
    
    \item \textbf{Cải thiện UX:}
    \begin{itemize}
        \item Thêm voice commands shortcuts
        \item Tutorials và onboarding bằng giọng nói
        \item Customizable voice preferences
    \end{itemize}
\end{enumerate}

\subsection{Trung hạn (6-12 tháng)}

\begin{enumerate}
    \item \textbf{Mobile Application:}
    \begin{itemize}
        \item Phát triển app iOS/Android native
        \item Tích hợp với accessibility features của mobile OS
        \item Offline mode với local TTS
    \end{itemize}
    
    \item \textbf{Thêm tính năng mới:}
    \begin{itemize}
        \item Object detection và navigation guidance
        \item Document scanning và OCR
        \item Real-time translation
        \item Smart reminders và scheduling
    \end{itemize}
    
    \item \textbf{Cải thiện AI:}
    \begin{itemize}
        \item Fine-tune model cho accessibility domain
        \item Personalization dựa trên user preferences
        \item Proactive suggestions và alerts
    \end{itemize}
\end{enumerate}

\subsection{Dài hạn (1-2 năm)}

\begin{enumerate}
    \item \textbf{Hardware Integration:}
    \begin{itemize}
        \item Tích hợp với smart glasses
        \item Wearable devices với haptic feedback
        \item IoT home automation control
    \end{itemize}
    
    \item \textbf{Community Platform:}
    \begin{itemize}
        \item Marketplace cho third-party plugins
        \item Community-driven feature requests
        \item Shared resources và tips
    \end{itemize}
    
    \item \textbf{Enterprise Solutions:}
    \begin{itemize}
        \item Accessibility compliance tools cho doanh nghiệp
        \item Training platform cho người khiếm thị
        \item Integration với workplace tools
    \end{itemize}
\end{enumerate}

\section{Bài học kinh nghiệm}

\subsection{Về quản lý dự án}

\begin{enumerate}
    \item \textbf{Phân công công việc rõ ràng:} Mỗi thành viên phụ trách một phần cụ thể (Frontend, Backend, AI Agent) giúp tăng hiệu quả
    
    \item \textbf{Communication thường xuyên:} Họp nhóm định kỳ để sync progress và giải quyết blockers
    
    \item \textbf{Documentation:} Viết document song song với code giúp các thành viên khác dễ hiểu và maintain
\end{enumerate}

\subsection{Về kỹ thuật}

\begin{enumerate}
    \item \textbf{Start simple:} Bắt đầu với MVP, sau đó iterate và improve
    
    \item \textbf{Test sớm với real users:} Feedback từ target users rất quan trọng để điều chỉnh hướng phát triển
    
    \item \textbf{Accessibility from day one:} Thiết kế accessible từ đầu dễ hơn nhiều so với retrofit sau này
\end{enumerate}

\subsection{Về nghiên cứu}

\begin{enumerate}
    \item \textbf{Empathy với users:} Hiểu sâu về pain points của người khiếm thị qua nhiều nguồn khác nhau
    
    \item \textbf{Study existing solutions:} Học hỏi từ những sản phẩm đã có trên thị trường
    
    \item \textbf{Iterate based on feedback:} Liên tục cải thiện dựa trên phản hồi
\end{enumerate}

\section{Lời kết}

Dự án BlindChat là nỗ lực của nhóm trong việc áp dụng các kiến thức về Tương tác Người-Máy để giải quyết một vấn đề thực tế: hỗ trợ người khiếm thị tiếp cận công nghệ. Mặc dù còn nhiều hạn chế, chúng tôi tin rằng đây là một bước đi đúng hướng trong việc xây dựng công nghệ inclusive và accessible cho tất cả mọi người.

Với sự phát triển không ngừng của AI, đặc biệt là các Large Language Models và multimodal AI, khả năng hỗ trợ người khuyết tật sẽ ngày càng được cải thiện. BlindChat có thể là nền tảng để phát triển thêm nhiều tính năng hữu ích hơn trong tương lai.

Chúng tôi hy vọng rằng dự án này sẽ góp phần nhỏ vào việc thu hẹp khoảng cách số (digital divide) và giúp người khiếm thị có thể độc lập hơn trong việc sử dụng công nghệ hàng ngày.

\vspace{1cm}
\begin{center}
\textit{``Technology is best when it brings people together.''}

\textit{--- Matt Mullenweg}
\end{center}

 % Kết luận và Hướng phát triển

% --- TÀI LIỆU THAM KHẢO ---
\newpage
\addcontentsline{toc}{chapter}{TÀI LIỆU THAM KHẢO}
\begin{center}
    {\fontsize{16}{19}\selectfont \textbf{TÀI LIỆU THAM KHẢO}}
    \vspace{1cm}
\end{center}

\textbf{Tài liệu về HCI và Accessibility:}
\begin{enumerate}
    \item Nielsen, J. (1994). \textit{10 Usability Heuristics for User Interface Design}. Nielsen Norman Group.
    
    \item W3C. (2018). \textit{Web Content Accessibility Guidelines (WCAG) 2.1}. World Wide Web Consortium. URL: \url{https://www.w3.org/WAI/WCAG21/}
    
    \item W3C. (2017). \textit{WAI-ARIA Authoring Practices 1.1}. World Wide Web Consortium. URL: \url{https://www.w3.org/WAI/ARIA/apg/}
    
    \item World Health Organization. (2019). \textit{World Report on Vision}. WHO.
\end{enumerate}

\textbf{Tài liệu về Công nghệ AI:}
\begin{enumerate}
    \setcounter{enumi}{4}
    \item Vaswani, A. et al. (2017). \textit{Attention Is All You Need}. NeurIPS.
    
    \item OpenAI. (2024). \textit{GPT-4 Technical Report}. URL: \url{https://openai.com/research/gpt-4}
    
    \item Microsoft Azure. (2024). \textit{Azure Speech Services Documentation}. URL: \url{https://docs.microsoft.com/azure/cognitive-services/speech-service/}
\end{enumerate}

\textbf{Tài liệu về Công nghệ Web:}
\begin{enumerate}
    \setcounter{enumi}{7}
    \item LiveKit. (2024). \textit{LiveKit Documentation}. URL: \url{https://docs.livekit.io/}
    
    \item Next.js. (2024). \textit{Next.js Documentation}. Vercel. URL: \url{https://nextjs.org/docs}
    
    \item Microsoft. (2024). \textit{ASP.NET Core Documentation}. URL: \url{https://docs.microsoft.com/aspnet/core/}
\end{enumerate}

\textbf{Tài liệu về Accessibility Technologies:}
\begin{enumerate}
    \setcounter{enumi}{10}
    \item NV Access. (2024). \textit{NVDA User Guide}. URL: \url{https://www.nvaccess.org/}
    
    \item Apple. (2024). \textit{VoiceOver User Guide}. URL: \url{https://support.apple.com/guide/voiceover/}
    
    \item Be My Eyes. (2024). \textit{Be My AI}. URL: \url{https://www.bemyeyes.com/}
    
    \item Microsoft. (2024). \textit{Seeing AI}. URL: \url{https://www.microsoft.com/en-us/seeing-ai}
\end{enumerate}

\end{document}