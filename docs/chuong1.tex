% =========================================
% CHƯƠNG 1: MỞ ĐẦU
% =========================================
\chapter{MỞ ĐẦU}

\section{Đặt vấn đề}

Theo Tổ chức Y tế Thế giới (WHO), trên toàn cầu có khoảng 2.2 tỷ người gặp vấn đề về thị lực, trong đó có ít nhất 1 tỷ người có tình trạng suy giảm thị lực có thể phòng ngừa hoặc chưa được điều trị. Tại Việt Nam, ước tính có khoảng 2 triệu người khiếm thị và mù lòa, chiếm khoảng 2\% dân số. Đây là nhóm đối tượng gặp nhiều khó khăn trong việc tiếp cận thông tin và sử dụng các thiết bị công nghệ hiện đại.

Trong bối cảnh chuyển đổi số đang diễn ra mạnh mẽ, việc tương tác với máy tính và các thiết bị số trở thành nhu cầu thiết yếu của mọi người. Tuy nhiên, phần lớn các ứng dụng và giao diện người dùng hiện nay được thiết kế chủ yếu dựa trên tương tác thị giác (visual interaction) như: màn hình, bàn phím, chuột, touch screen. Điều này tạo ra rào cản lớn đối với người khiếm thị trong việc:

\begin{itemize}
    \item Đọc và xử lý tài liệu số (PDF, Word, email)
    \item Truy cập thông tin trên web và ứng dụng
    \item Nhận diện môi trường xung quanh thông qua camera
    \item Điều khiển các thiết bị và ứng dụng một cách độc lập
\end{itemize}

Các công cụ hỗ trợ hiện có như screen reader (NVDA, JAWS, VoiceOver) tuy hữu ích nhưng vẫn yêu cầu người dùng phải ghi nhớ nhiều phím tắt phức tạp và không thể xử lý các tác vụ yêu cầu ``nhìn'' như mô tả hình ảnh hoặc nhận diện đối tượng.

Với sự phát triển vượt bậc của trí tuệ nhân tạo (AI), đặc biệt là các mô hình ngôn ngữ lớn (Large Language Models - LLM) và xử lý giọng nói, giờ đây có thể xây dựng một hệ thống tương tác hoàn toàn bằng giọng nói, cho phép người khiếm thị giao tiếp với máy tính một cách tự nhiên như đang nói chuyện với một người trợ lý thực sự.

\section{Hiện trạng}

\subsection{Tình hình sử dụng công nghệ của người khiếm thị}

Hiện nay, người khiếm thị tại Việt Nam chủ yếu sử dụng các công cụ hỗ trợ sau để tương tác với thiết bị số:

\begin{enumerate}
    \item \textbf{Screen Reader (Trình đọc màn hình):} Các phần mềm như NVDA (miễn phí), JAWS (có phí), VoiceOver (tích hợp sẵn trên macOS/iOS) giúp đọc nội dung trên màn hình thành giọng nói. Tuy nhiên, người dùng cần phải học và ghi nhớ hàng trăm phím tắt để điều khiển.
    
    \item \textbf{Braille Display:} Thiết bị hiển thị chữ nổi Braille kết nối với máy tính. Chi phí cao (từ 1,000 - 5,000 USD) khiến phần lớn người khiếm thị tại Việt Nam không thể tiếp cận.
    
    \item \textbf{Voice Assistant cơ bản:} Siri, Google Assistant, Alexa cung cấp tương tác giọng nói nhưng chức năng hạn chế, chủ yếu tập trung vào các tác vụ đơn giản như đặt báo thức, tra cứu thời tiết.
\end{enumerate}

\subsection{Những thách thức chính}

\begin{itemize}
    \item \textbf{Rào cản ngôn ngữ:} Hầu hết các công cụ hỗ trợ người khiếm thị được phát triển bằng tiếng Anh, hỗ trợ tiếng Việt còn hạn chế.
    
    \item \textbf{Đường cong học tập cao:} Việc sử dụng screen reader đòi hỏi người dùng phải ghi nhớ nhiều phím tắt phức tạp.
    
    \item \textbf{Không xử lý được nội dung hình ảnh:} Các công cụ hiện tại không thể ``nhìn'' và mô tả hình ảnh, biểu đồ, hoặc môi trường xung quanh.
    
    \item \textbf{Thiếu tính cá nhân hóa:} Không có khả năng hiểu ngữ cảnh và đưa ra phản hồi phù hợp với từng người dùng.
\end{itemize}

\section{Các giải pháp sẵn có}

\subsection{Giải pháp trong nước}

Tại Việt Nam, các giải pháp hỗ trợ người khiếm thị còn khá hạn chế:

\begin{itemize}
    \item \textbf{Ứng dụng Sách nói:} Một số ứng dụng như ``Sách nói Việt Nam'' cung cấp audiobook cho người khiếm thị, nhưng nội dung còn hạn chế và không có tính tương tác.
    
    \item \textbf{NVDA với tiếng Việt:} Phần mềm NVDA được cộng đồng Việt hóa nhưng chất lượng giọng đọc chưa tự nhiên và thiếu nhiều tính năng nâng cao.
    
    \item \textbf{Các dự án nghiên cứu:} Một số đề tài nghiên cứu tại các trường đại học về công nghệ hỗ trợ người khiếm thị nhưng chưa được thương mại hóa và triển khai rộng rãi.
\end{itemize}

\subsection{Giải pháp quốc tế}

\begin{itemize}
    \item \textbf{Be My Eyes:} Ứng dụng kết nối người khiếm thị với tình nguyện viên sáng mắt qua video call để được hỗ trợ mô tả hình ảnh. Gần đây tích hợp GPT-4 Vision để tự động mô tả hình ảnh.
    
    \item \textbf{Seeing AI (Microsoft):} Ứng dụng sử dụng AI để nhận diện và mô tả đối tượng, đọc văn bản, nhận diện khuôn mặt. Chỉ hỗ trợ tiếng Anh và một số ngôn ngữ phổ biến.
    
    \item \textbf{Aira:} Dịch vụ cung cấp ``agent'' là người thật để hỗ trợ người khiếm thị 24/7 qua camera điện thoại. Chi phí cao (99-199 USD/tháng).
    
    \item \textbf{OrCam MyEye:} Thiết bị đeo gắn trên kính, sử dụng AI để đọc văn bản và nhận diện khuôn mặt. Chi phí rất cao (khoảng 4,500 USD).
\end{itemize}

\section{Hạn chế của các giải pháp hiện tại}

Qua phân tích các giải pháp sẵn có, chúng tôi nhận thấy những hạn chế chính sau:

\begin{enumerate}
    \item \textbf{Chi phí cao:} Các giải pháp tiên tiến như OrCam MyEye, Aira có chi phí vượt quá khả năng tài chính của đa số người khiếm thị tại Việt Nam.
    
    \item \textbf{Hạn chế ngôn ngữ:} Phần lớn không hỗ trợ hoặc hỗ trợ tiếng Việt kém.
    
    \item \textbf{Phụ thuộc vào thiết bị chuyên dụng:} Nhiều giải pháp yêu cầu thiết bị phần cứng riêng, không tận dụng được thiết bị sẵn có của người dùng (laptop, smartphone).
    
    \item \textbf{Thiếu tích hợp:} Mỗi ứng dụng giải quyết một vấn đề riêng lẻ, người dùng phải chuyển đổi giữa nhiều ứng dụng để hoàn thành công việc.
    
    \item \textbf{Không có context và memory:} Các voice assistant hiện tại không nhớ ngữ cảnh hội thoại trước đó, mỗi lần tương tác là một phiên mới hoàn toàn.
\end{enumerate}

\section{Mục tiêu và Phạm vi đề tài}

\subsection{Mục tiêu}

Dự án \textbf{BlindChat} được phát triển với các mục tiêu sau:

\begin{enumerate}
    \item \textbf{Xây dựng giao diện giọng nói hoàn toàn:} Cho phép người khiếm thị tương tác với hệ thống 100\% bằng giọng nói, không cần sử dụng bàn phím hay chuột.
    
    \item \textbf{Tích hợp AI đa phương thức:} Kết hợp xử lý ngôn ngữ tự nhiên (NLP), nhận dạng giọng nói (STT), tổng hợp giọng nói (TTS), và xử lý hình ảnh (Vision) trong một hệ thống thống nhất.
    
    \item \textbf{Hỗ trợ đọc tài liệu thông minh:} Cho phép người dùng yêu cầu đọc hoặc tóm tắt các file PDF bằng giọng nói.
    
    \item \textbf{Mô tả hình ảnh từ camera:} Sử dụng AI Vision để mô tả những gì camera nhìn thấy, giúp người khiếm thị ``nhìn'' môi trường xung quanh.
    
    \item \textbf{Duy trì ngữ cảnh hội thoại:} Lưu trữ và sử dụng lịch sử hội thoại để AI hiểu context và đưa ra phản hồi chính xác hơn.
    
    \item \textbf{Thiết kế Accessibility-first:} Áp dụng các nguyên tắc thiết kế accessible từ đầu, đảm bảo trải nghiệm tốt nhất cho người khiếm thị.
\end{enumerate}

\subsection{Phạm vi}

\textbf{Trong phạm vi đề tài:}
\begin{itemize}
    \item Phát triển ứng dụng web với giao diện giọng nói real-time
    \item Tích hợp AI Agent với các chức năng: đọc file, mô tả hình ảnh, điều khiển UI
    \item Xây dựng hệ thống backend lưu trữ lịch sử hội thoại
    \item Hỗ trợ tiếng Anh (có thể mở rộng tiếng Việt trong tương lai)
    \item Chạy trên trình duyệt web (Chrome, Firefox, Edge)
\end{itemize}

\textbf{Ngoài phạm vi đề tài:}
\begin{itemize}
    \item Phát triển ứng dụng mobile native (iOS/Android)
    \item Tích hợp thiết bị phần cứng chuyên dụng
    \item Hỗ trợ đa ngôn ngữ (ngoài tiếng Anh)
    \item Tích hợp với các hệ thống enterprise
\end{itemize}

\section{Đóng góp của đề tài}

Đề tài mang lại những đóng góp sau:

\begin{enumerate}
    \item \textbf{Về mặt khoa học:}
    \begin{itemize}
        \item Nghiên cứu và áp dụng các nguyên tắc thiết kế Human-Computer Interaction (HCI) cho người khiếm thị
        \item Tích hợp nhiều công nghệ AI tiên tiến trong một hệ thống thống nhất
        \item Đề xuất kiến trúc hệ thống voice-first application
    \end{itemize}
    
    \item \textbf{Về mặt thực tiễn:}
    \begin{itemize}
        \item Cung cấp một công cụ miễn phí, dễ tiếp cận cho người khiếm thị
        \item Giảm rào cản công nghệ, giúp người khiếm thị độc lập hơn trong công việc và cuộc sống
        \item Mã nguồn mở, có thể được cộng đồng phát triển và mở rộng
    \end{itemize}
\end{enumerate}

\section{Cấu trúc báo cáo}

Báo cáo được tổ chức thành các chương như sau:

\begin{itemize}
    \item \textbf{Chương 1 - Mở đầu:} Trình bày đặt vấn đề, hiện trạng, các giải pháp sẵn có, mục tiêu và phạm vi đề tài.
    
    \item \textbf{Chương 2 - Nghiên cứu người dùng:} Phân tích đối tượng người dùng, xây dựng User Persona, User Journey Map và xác định các yêu cầu hệ thống.
    
    \item \textbf{Chương 3 - Cơ sở lý thuyết và Công nghệ:} Trình bày nền tảng lý thuyết về HCI, Accessibility và các công nghệ sử dụng.
    
    \item \textbf{Chương 4 - Thiết kế hệ thống:} Mô tả kiến trúc, thiết kế giao diện và luồng tương tác.
    
    \item \textbf{Chương 5 - Triển khai và Kết quả:} Trình bày quá trình triển khai, demo và đánh giá hệ thống.
    
    \item \textbf{Chương 6 - Kết luận:} Tổng kết kết quả đạt được, hạn chế và hướng phát triển.
\end{itemize}

