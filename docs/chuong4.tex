% =========================================
% CHƯƠNG 4: THIẾT KẾ HỆ THỐNG
% =========================================
\chapter{THIẾT KẾ HỆ THỐNG}

Chương này trình bày chi tiết kiến trúc hệ thống, thiết kế cơ sở dữ liệu, thiết kế API và thiết kế giao diện người dùng của ứng dụng BlindChat.

\section{Kiến trúc tổng thể}

\subsection{Tổng quan kiến trúc}

Hệ thống BlindChat được thiết kế theo kiến trúc microservices với 3 thành phần chính:

\begin{figure}[H]
    \centering
    \begin{verbatim}
    +-------------------------------------------------------------+
    |                         Frontend                            |
    |              (Next.js 15 + React 19 + LiveKit)              |
    |         Voice Interface, Video Streaming, Chat UI           |
    +--------------------------+----------------------------------+
                               |
                   +-----------+-----------+
                   |                       |
                   v                       v
    +----------------------+   +--------------------------------+
    |      AI Agent        |   |         Backend API            |
    |  (Python + LiveKit)  |   |   (ASP.NET Core 9 + SQL)       |
    |                      |   |                                |
    | * Voice Assistant    |<--|  * User Authentication         |
    | * OpenAI GPT-4o-mini |   |  * Conversation History        |
    | * Azure STT/TTS      |   |  * Message Storage             |
    | * Vision Processing  |   |                                |
    +----------------------+   +--------------------------------+
    \end{verbatim}
    \caption{Kiến trúc tổng thể hệ thống BlindChat}
\end{figure}

\subsection{Luồng dữ liệu}

\subsubsection{Luồng Voice Interaction}

\begin{enumerate}
    \item User nói vào microphone $\rightarrow$ Frontend capture audio
    \item Audio được stream qua LiveKit Room đến AI Agent
    \item AI Agent sử dụng Azure STT để chuyển speech thành text
    \item Text được gửi đến OpenAI GPT-4o-mini để xử lý
    \item GPT-4o-mini quyết định gọi tool hoặc trả lời trực tiếp
    \item Response text được gửi qua Azure TTS để tạo audio
    \item Audio response được stream về Frontend qua LiveKit
    \item User nghe phản hồi
\end{enumerate}

\subsubsection{Luồng lưu trữ Conversation}

\begin{enumerate}
    \item Mỗi tin nhắn (user/agent) được lưu vào ConversationCache
    \item Khi đủ 5 cặp tin nhắn, cache tự động flush lên Backend
    \item Backend lưu messages vào SQL Server
    \item Khi user reconnect, Agent lấy lịch sử từ Backend để load context
\end{enumerate}

\section{Thiết kế AI Agent}

\subsection{Class Diagram}

\begin{figure}[H]
    \centering
    \begin{verbatim}
    +-----------------------------------------------------+
    |                    Assistant                        |
    |                   (extends Agent)                   |
    +-----------------------------------------------------+
    | - context_pairs: List[Dict]                         |
    | - cache: ConversationCache                          |
    +-----------------------------------------------------+
    | + update_context(username: str)                     |
    | + get_current_date_and_time() -> str                |
    | + process_file_request(user_input: str) -> str      |
    | + describe_camera_view(user_input: str) -> str      |
    | + control_ui_device(target, status) -> str          |
    +-----------------------------------------------------+
                              |
                              | uses
                              v
    +-----------------------------------------------------+
    |                ConversationCache                    |
    +-----------------------------------------------------+
    | - username: str                                     |
    | - pairs_to_flush: int                               |
    | - _pending_messages: List[Dict]                     |
    | - _token: Optional[str]                             |
    +-----------------------------------------------------+
    | + add_user_message(content: str)                    |
    | + add_agent_message(content: str)                   |
    | + flush() -> List[Dict]                             |
    | + get_last_n_pairs(n: int) -> List[Dict]            |
    | + get_history_messages() -> List[Dict]              |
    +-----------------------------------------------------+
    \end{verbatim}
    \caption{Class Diagram của AI Agent}
\end{figure}

\subsection{Function Tools}

AI Agent được trang bị 4 function tools:

\begin{table}[H]
    \centering
    \renewcommand{\arraystretch}{1.5}
    \footnotesize
    \begin{tabular}{|>{\raggedright\arraybackslash}p{3.8cm}|>{\raggedright\arraybackslash}p{4.2cm}|>{\raggedright\arraybackslash}p{5.5cm}|}
        \hline
        \textbf{Tool Name} & \textbf{Trigger Phrases} & \textbf{Chức năng} \\
        \hline
        \texttt{get\_current\_} \texttt{date\_and\_time} & ``What time is it?'', ``What's the date?'' & Trả về ngày giờ hiện tại \\
        \hline
        \texttt{process\_file\_} \texttt{request} & ``Read my PDF'', ``Summarize the report'' & Đọc và tóm tắt file PDF từ Downloads \\
        \hline
        \texttt{describe\_} \texttt{camera\_view} & ``What do you see?'', ``Describe this'' & Mô tả hình ảnh từ camera \\
        \hline
        \texttt{control\_ui\_} \texttt{device} & ``Turn on camera'', ``Mute microphone'' & Điều khiển UI (camera, mic, chat) \\
        \hline
    \end{tabular}
    \caption{Function Tools của AI Agent}
\end{table}

\subsection{Request Routing}

Hệ thống sử dụng LLM để route user request:

\begin{verbatim}
RequestType:
  - request_type: "read raw text" | "read file and summary" | "unsupported"
  - confidence_score: float (0-1)
  - description: str
  - file_name: Optional[str]
  - nth_file: Optional[int]
\end{verbatim}

Luồng routing:
\begin{enumerate}
    \item User input + conversation context $\rightarrow$ Router LLM
    \item Router phân loại request type với confidence score
    \item Nếu confidence $\geq$ 0.7: thực hiện action tương ứng
    \item Nếu confidence $<$ 0.7: trả về ``unsupported''
\end{enumerate}

\section{Thiết kế Backend API}

\subsection{Database Schema}

\begin{figure}[H]
    \centering
    \begin{verbatim}
    +------------------+
    |      User        |
    +------------------+
    | Id: string (PK)  |
    | UserName: string |
    | (IdentityUser)   |
    +--------+---------+
             | 1
             |
             | 1
    +--------v-----------------+
    |   ConversationHistory    |
    +--------------------------+
    | Id: int (PK)             |
    | UserId: string (FK)      |
    | CreatedAt: DateTime      |
    +--------+-----------------+
             | 1
             |
             | *
    +--------v---------+
    |     Message      |
    +------------------+
    | Id: int (PK)     |
    | ConversationId   |
    | SenderType: int  |
    | Content: string  |
    | CreatedAt        |
    +------------------+
    
    SenderType: 0 = User, 1 = Bot
    \end{verbatim}
    \caption{Database Schema}
\end{figure}

\subsection{API Endpoints}

\subsubsection{Account Controller}

\begin{table}[H]
    \centering
    \renewcommand{\arraystretch}{1.3}
    \begin{tabular}{|l|l|p{5cm}|p{4cm}|}
        \hline
        \textbf{Method} & \textbf{Endpoint} & \textbf{Request Body} & \textbf{Response} \\
        \hline
        POST & /api/account/register & \texttt{\{username: string\}} & User + JWT Token \\
        \hline
        POST & /api/account/login & \texttt{\{username: string\}} & User + JWT Token \\
        \hline
    \end{tabular}
    \caption{Account API Endpoints}
\end{table}

\subsubsection{Message Controller}

\begin{table}[H]
    \centering
    \renewcommand{\arraystretch}{1.3}
    \begin{tabular}{|l|l|p{5cm}|p{4cm}|}
        \hline
        \textbf{Method} & \textbf{Endpoint} & \textbf{Request Body} & \textbf{Response} \\
        \hline
        POST & /api/messages & List of CreateMessageDto & List of MessageDto \\
        \hline
    \end{tabular}
    \caption{Message API Endpoint}
\end{table}

\textbf{CreateMessageDto:}
\begin{verbatim}
{
    "senderType": 0 | 1,
    "content": "string",
    "createdAt": "2025-01-01T00:00:00Z"
}
\end{verbatim}

\subsubsection{Conversation History Controller}

\begin{table}[H]
    \centering
    \renewcommand{\arraystretch}{1.3}
    \begin{tabular}{|l|l|p{5cm}|p{4cm}|}
        \hline
        \textbf{Method} & \textbf{Endpoint} & \textbf{Query Params} & \textbf{Response} \\
        \hline
        GET & /api/conversation-history & \texttt{limit: int} & ConversationHistoryDto \\
        \hline
    \end{tabular}
    \caption{Conversation History API Endpoint}
\end{table}

\subsection{Authentication Flow}

\begin{enumerate}
    \item AI Agent gọi POST /api/account/login với username
    \item Backend trả về JWT token
    \item Agent lưu token và sử dụng trong header: \texttt{Authorization: Bearer <token>}
    \item Token có thời hạn 1 giờ, agent tự động refresh khi hết hạn
\end{enumerate}

\section{Thiết kế Frontend}

\subsection{Component Architecture}

\begin{verbatim}
App
|-- SessionProvider (Context: LiveKit Room, AppConfig)
|   +-- ViewController
|       |-- WelcomeView (Before connect)
|       |   +-- StartButton
|       +-- SessionView (After connect)
|           |-- TileLayout (Video tiles)
|           |-- ChatTranscript (Chat history)
|           +-- AgentControlBar
|               |-- TrackToggle (Mic)
|               |-- TrackToggle (Camera)
|               |-- ChatInput
|               +-- LeaveButton
\end{verbatim}

\subsection{State Management}

Frontend sử dụng React Context và LiveKit hooks:

\begin{itemize}
    \item \textbf{SessionContext:} Quản lý connection state, app config
    \item \textbf{useRoomContext:} Access LiveKit Room object
    \item \textbf{useChat:} Manage chat messages
    \item \textbf{useTranscriptions:} Real-time transcriptions
    \item \textbf{useChatMessages:} Merge transcriptions + chat messages
\end{itemize}

\subsection{Connection Flow}

\begin{enumerate}
    \item User click ``Start Call''
    \item Frontend gọi POST /api/connection-details
    \item Server tạo LiveKit access token với:
    \begin{itemize}
        \item Room name: \texttt{voice\_assistant\_room\_\{random\}}
        \item Participant identity: user identity
        \item Grants: roomJoin, canPublish, canSubscribe
    \end{itemize}
    \item Frontend connect đến LiveKit server với token
    \item AI Agent tự động join room khi có participant
\end{enumerate}

\subsection{Data Channel Communication}

Để điều khiển UI từ AI Agent, sử dụng LiveKit Data Channel:

\textbf{Agent gửi command:}
\begin{verbatim}
payload = {
    "type": "control_camera" | "control_microphone" | "control_chat",
    "status": "on" | "off"
}
await room.local_participant.publish_data(
    json.dumps(payload),
    destination_identities=[participant.identity]
)
\end{verbatim}

\textbf{Frontend nhận và xử lý:}
\begin{verbatim}
room.on('dataReceived', (payload, participant) => {
    const command = JSON.parse(payload);
    if (command.type === 'control_camera') {
        toggleCamera(command.status === 'on');
    }
    // ...
});
\end{verbatim}

\section{Thiết kế giao diện}

\subsection{Nguyên tắc thiết kế Voice-First UI}

\begin{enumerate}
    \item \textbf{Minimal Visual Elements:} Giao diện tối giản, không gây phân tán
    \item \textbf{High Contrast:} Độ tương phản cao cho người low vision
    \item \textbf{Large Touch Targets:} Nút bấm lớn, dễ tương tác
    \item \textbf{Audio Feedback:} Mọi hành động đều có phản hồi âm thanh
    \item \textbf{Screen Reader Compatible:} Tương thích hoàn toàn với VoiceOver/NVDA
\end{enumerate}

\subsection{Layout Design}

\subsubsection{Welcome View}

\begin{verbatim}
+-----------------------------------------+
|                                         |
|              [Logo BlindChat]           |
|                                         |
|        "BlindChat Voice Agent"          |
|   "A voice agent support blind users"   |
|                                         |
|          +-----------------+            |
|          |   Start Call    |            |
|          +-----------------+            |
|                                         |
|            [Theme Toggle]               |
|                                         |
+-----------------------------------------+
\end{verbatim}

\subsubsection{Session View}

\begin{verbatim}
+-----------------------------------------+
|  +---------------------------------+    |
|  |                                 |    |
|  |      Video Tile (Agent/User)    |    |
|  |                                 |    |
|  +---------------------------------+    |
|                                         |
|  +---------------------------------+    |
|  | Chat Transcript                 |    |
|  | User: "Read my PDF"             |    |
|  | Agent: "Reading file..."        |    |
|  +---------------------------------+    |
|                                         |
|  +---------------------------------+    |
|  | [Mic] [Camera] [Chat] [Leave]   |    |
|  | [          Text Input         ] |    |
|  +---------------------------------+    |
+-----------------------------------------+
\end{verbatim}

\subsection{Accessibility Features}

\begin{table}[H]
    \centering
    \renewcommand{\arraystretch}{1.3}
    \begin{tabular}{|p{4cm}|p{10cm}|}
        \hline
        \textbf{Feature} & \textbf{Implementation} \\
        \hline
        Keyboard Navigation & Tất cả elements có thể focus bằng Tab \\
        \hline
        ARIA Labels & Mọi button và control đều có aria-label \\
        \hline
        Screen Reader & Semantic HTML, role attributes \\
        \hline
        Color Contrast & Tối thiểu 4.5:1 ratio \\
        \hline
        Focus Indicators & Visible focus ring cho keyboard users \\
        \hline
        Responsive Text & Hỗ trợ zoom đến 200\% \\
        \hline
        Dark Mode & Giảm strain cho người low vision \\
        \hline
    \end{tabular}
    \caption{Accessibility Features}
\end{table}

\section{Sequence Diagrams}

\subsection{Đọc file PDF}

\begin{verbatim}
User          Frontend      LiveKit      AI Agent       Backend
 |               |             |            |              |
 |--"Read PDF"-->|             |            |              |
 |               |--Audio----->|            |              |
 |               |             |--Stream--->|              |
 |               |             |            |--STT         |
 |               |             |            |--GPT-4o------|
 |               |             |            |  (tool call) |
 |               |             |            |--Read file   |
 |               |             |            |--Summarize   |
 |               |             |            |--TTS         |
 |               |             |<--Stream---|              |
 |               |<--Audio-----|            |              |
 |<--Response----|             |            |              |
\end{verbatim}

\subsection{Mô tả camera}

\begin{verbatim}
User          Frontend      LiveKit      AI Agent      OpenAI Vision
 |               |             |            |              |
 |--"What see?"->|             |            |              |
 |               |--Audio+Video>|           |              |
 |               |             |--Stream--->|              |
 |               |             |            |--Capture frame
 |               |             |            |--Base64 encode
 |               |             |            |------------->|
 |               |             |            |<--Description|
 |               |             |            |--TTS         |
 |               |             |<--Stream---|              |
 |<--Description-|             |            |              |
\end{verbatim}

\section{Kết luận chương}

Chương này đã trình bày chi tiết thiết kế hệ thống BlindChat bao gồm:

\begin{itemize}
    \item Kiến trúc tổng thể với 3 thành phần: Frontend, Backend, AI Agent
    \item Thiết kế AI Agent với 4 function tools
    \item Thiết kế Backend API với authentication và message storage
    \item Thiết kế Frontend với component architecture và state management
    \item Thiết kế giao diện tuân thủ accessibility guidelines
    \item Sequence diagrams cho các use case chính
\end{itemize}

Thiết kế này đảm bảo hệ thống có thể mở rộng, bảo trì dễ dàng và đáp ứng các yêu cầu accessibility cho người khiếm thị.
